%% Generated by Sphinx.
\def\sphinxdocclass{report}
\documentclass[a4paper,10pt,english]{sphinxmanual}
\ifdefined\pdfpxdimen
   \let\sphinxpxdimen\pdfpxdimen\else\newdimen\sphinxpxdimen
\fi \sphinxpxdimen=.75bp\relax

\PassOptionsToPackage{warn}{textcomp}
\usepackage[utf8]{inputenc}
\ifdefined\DeclareUnicodeCharacter
 \ifdefined\DeclareUnicodeCharacterAsOptional
  \DeclareUnicodeCharacter{"00A0}{\nobreakspace}
  \DeclareUnicodeCharacter{"2500}{\sphinxunichar{2500}}
  \DeclareUnicodeCharacter{"2502}{\sphinxunichar{2502}}
  \DeclareUnicodeCharacter{"2514}{\sphinxunichar{2514}}
  \DeclareUnicodeCharacter{"251C}{\sphinxunichar{251C}}
  \DeclareUnicodeCharacter{"2572}{\textbackslash}
 \else
  \DeclareUnicodeCharacter{00A0}{\nobreakspace}
  \DeclareUnicodeCharacter{2500}{\sphinxunichar{2500}}
  \DeclareUnicodeCharacter{2502}{\sphinxunichar{2502}}
  \DeclareUnicodeCharacter{2514}{\sphinxunichar{2514}}
  \DeclareUnicodeCharacter{251C}{\sphinxunichar{251C}}
  \DeclareUnicodeCharacter{2572}{\textbackslash}
 \fi
\fi
\usepackage{cmap}
\usepackage[T1]{fontenc}
\usepackage{amsmath,amssymb,amstext}
\usepackage{babel}
\usepackage{times}
\usepackage[Sonny]{fncychap}
\usepackage{sphinx}

\usepackage{geometry}

% Include hyperref last.
\usepackage{hyperref}
% Fix anchor placement for figures with captions.
\usepackage{hypcap}% it must be loaded after hyperref.
% Set up styles of URL: it should be placed after hyperref.
\urlstyle{same}
\addto\captionsenglish{\renewcommand{\contentsname}{目录}}

\addto\captionsenglish{\renewcommand{\figurename}{图}}
\addto\captionsenglish{\renewcommand{\tablename}{表}}
\addto\captionsenglish{\renewcommand{\literalblockname}{列表}}

\addto\captionsenglish{\renewcommand{\literalblockcontinuedname}{续上页}}
\addto\captionsenglish{\renewcommand{\literalblockcontinuesname}{continues on next page}}

\addto\extrasenglish{\def\pageautorefname{页}}

\setcounter{tocdepth}{1}


\hypersetup{unicode=true}
\usepackage{CJKutf8}
\DeclareUnicodeCharacter{00A0}{\nobreakspace}
\DeclareUnicodeCharacter{2203}{\ensuremath{\exists}}
\DeclareUnicodeCharacter{2286}{\ensuremath{\subseteq}}
\DeclareUnicodeCharacter{2713}{x}
\DeclareUnicodeCharacter{27FA}{\ensuremath{\Longleftrightarrow}}
\DeclareUnicodeCharacter{221A}{\ensuremath{\sqrt{}}}
\DeclareUnicodeCharacter{221B}{\ensuremath{\sqrt[3]{}}}
\DeclareUnicodeCharacter{2295}{\ensuremath{\oplus}}
\DeclareUnicodeCharacter{2297}{\ensuremath{\otimes}}
\begin{CJK}{UTF8}{gbsn}
\AtEndDocument{\end{CJK}}


\title{CUPTGuide Documentation}
\date{2019 年 12 月 08 日}
\release{}
\author{asd1dsa}
\newcommand{\sphinxlogo}{\vbox{}}
\renewcommand{\releasename}{}
\makeindex

\begin{document}

\maketitle
\sphinxtableofcontents
\phantomsection\label{\detokenize{index::doc}}


最近的重要更新发布于2019年11月27日,更新了 \sphinxstylestrong{学术报告} 部分。

如有反馈意见请从右上角链接前往Github的Issues页面,也欢迎fork。此外,在Github仓库中的source文件夹中能找到一部分绘图用的代码。


\chapter{前言}
\label{\detokenize{1. Preface:by-asd1dsa}}\label{\detokenize{1. Preface:id1}}\label{\detokenize{1. Preface::doc}}
我代表学校参加了第八届大学生物理学术竞赛(CUPT),并且在下一届参赛的同时提供了其他方面的各种帮助,对这个竞赛相对还算比较了解,并且本身算是一个物理爱好者,在此为各位介绍它。

这个比赛首先是一个 \sphinxstylestrong{学术竞赛} ,而非一般学科竞赛中以做题为考核方式。涉及的主要任务是研究,其次是展示。研究的思路和成果虽说通常与真正的科学研究尚有差距,但至少算是一次 \sphinxstylestrong{科研的体验之旅} 。

这个比赛通常是低年级本科生参加,高中毕业后至多不到两年的学生不太可能积累有较充分的知识储备,一切所需的理论知识和研究方法都需要自己钻研、磨练。经过这样一个 \sphinxstylestrong{在实践项目中学习} 的过程,对理论知识与实际的联系会更容易,也能培养提出问题、解决问题的重要思维习惯。

我认为,它最大的意义是作为一个尝试的机会。高中生进入大学就开始修习专门的专业知识,但往往在并不了解学科的情况下就茫然学习,通常无法建立真实的兴趣。要寻找到自己的兴趣,需要先去尝试各种各样的可能适合自己的活动,这样才能了解这些活动,自然地建立兴趣,进而拥有追求,便不再会为人生的抉择而犯难。而我认为,这个比赛就是科研这条路线的尝试的一个好机会。

值得一提的是,虽然其名为物理学术竞赛,实际上它对各理工科专业的学生都有很好的锻炼作用,反倒由于低年级学生难以积累过多物理知识而使得物理学理论中特有的经典思想方法不太会出现,许多物理系的学生也会因为理论知识的学习任务太重而无心参与这个比赛,于是工科思维在这个比赛的研究中也占据了相当的分量。不过这对于物理学也不是问题,物理学虽说是抽象的“自然哲学的数学原理”,但终归是一门科学,是讲究可证伪、讲究实际的,专注于解决问题的工程思维并不会成为学理论物理的障碍。

不过,我所说的这样的尝试的意义在于区分自己是不是能够对这个活动感兴趣,那就意味着你参加之前不可能知道自己做起来以后会不会感兴趣。

我的建议是:既然这样,那就来试试吧!

这和科研的一个特性很像:平时我们一但遇到了预料之外的、失去掌控的东西便会苦恼,但做研究是思路似乎就是“要迎难而上”,因为预料之外的、失去掌控的事物就是研究者所要寻找、理解的。

在兴趣这个问题上也是类似的,迎难而上会是一个不错的做法。虽然你参加之前不可能知道自己做起来以后会不会感兴趣,但为了区分自己是不是能够对这个活动感兴趣就需要做这样的尝试,就来试试吧。

你确实很容易就可以从中寻找到乐趣。这种乐趣和题目的设计也有关系:竞赛题目指定要研究的现象大多源于生活,往往我们早已遇到过它们但未曾深入思考、仔细调查。涉及的领域距离我们也并不遥远,不是天上的脉冲星也不是微观的纠缠粒子,而是身边的流体、光、声音等等。这些现象往往乍一看匪夷所思,但稍作思考、测试,便能收集到更多的信息,我们就可能提出一些假想来解释它,经过一些区分能力更强的实验检验以后,我们最终可以相信其中一个假想作为答案,这时就会有柳暗花明之感洋溢而出。

此外,如果争取到代表学校参赛的资格,就能进行另一项重要的活动——与其他高校的学术辩论。值得一提的是,学术辩论不同于普通的辩论,其目的到底还是清晰地展示和客观地评价一个研究工作,从而探讨更好的改进方案,其首要原则是 \sphinxstylestrong{实事求是} ,不能诡辩。赛场上面向的不是对手而是伙伴,共同的目标是弄清楚现象的物理机制,而真相只有一个,不存在两方观点相矛盾但又都有道理的情况,讨论的目的是找到那个真相而非言语上取胜,相互的质疑只是为了帮助消除个人思维的局限性罢了。

\sphinxstylestrong{向他人展示自己的研究成果} 在研究工作中也是不可缺少的。你能让别人认为你的研究有价值,才能得到各方面的支持。此外,展示的过程也是整理的过程,在回顾过去工作的过程中能加深自己对问题的理解。学术辩论过程中所涉及的对他人研究成果的讨论、评价,这些同样对敏捷思维的训练和深刻理解的积累大有好处。


\chapter{竞赛简介}
\label{\detokenize{2. Intro_Tournament:id1}}\label{\detokenize{2. Intro_Tournament::doc}}
规则和建议都暂且简单介绍,规则详见官方 \sphinxstylestrong{竞赛指南} ,建议则会在接下来的章节细化。


\section{竞赛任务}
\label{\detokenize{2. Intro_Tournament:id2}}
竞赛要求一个队伍(3\textasciitilde{}5人)完成指定的研究任务,然后在赛场上扮演 \sphinxstylestrong{正方{[}Report{]}} 、 \sphinxstylestrong{反方{[}Opposite{]}} 、 \sphinxstylestrong{评论方{[}Review{]}} 三种角色。三种角色的任务分别是 \sphinxstylestrong{报告研究成果并与反方讨论以及回答反方和评论方的问题} 、 \sphinxstylestrong{总结正方研究的优缺点并且深入讨论改进方案或物理本质等} 、 \sphinxstylestrong{总结报告和讨论要点并且补充遗漏点} 。


\section{赛题}
\label{\detokenize{2. Intro_Tournament:id3}}
赛题即需要完成的研究内容,直接使用IYPT的赛题,一共17道,需要完成其中的15道及以上(地区赛要求稍低),如果放弃题目会有分数上的惩罚。赛题的内容一般包括两部分: \sphinxstylestrong{现象描述} 和 \sphinxstylestrong{任务指定} ,偶尔也有不这样构成的题目。


\subsection{常见的研究任务}
\label{\detokenize{2. Intro_Tournament:id4}}\begin{itemize}
\item {} 
原理解释: Explain …

\item {} 
创新制造: Construct …/Design …

\item {} 
研究参量对现象的影响: Investigate how … depends on …

\item {} 
最优化(为了最优化,一般也需要完成上一种任务): Optimize …

\item {} 
性能评价(许多时候被其他的任务所隐含)

\item {} 
寻找现象发生的条件: Determine the condition/parameters that …

\item {} 
自由选择: Investigate the phenomenon.

\end{itemize}


\section{竞赛主要相关资料来源}
\label{\detokenize{2. Intro_Tournament:id5}}\begin{itemize}
\item {} 
IYPT官方网站的档案,往年题目、比赛视频和研究资料:\sphinxurl{http://archive.iypt.org}

\item {} 
GYPT官方网站的指引,题目、视频、指引:\sphinxurl{https://gypt.org/aufgaben.html} (德文,建议翻译成英文看)

\item {} 
Reference Kit,某学者整理的参考资料,提供论文、视频的标题和链接: \sphinxurl{http://kit.ilyam.org}

\item {} 
STEM,加拿大学生团体,整理的参考资料,提供论文、视频的标题和链接:\sphinxurl{http://stemfellowship.org/iypt-2020-references-2}

\item {} 
哔哩哔哩弹幕视频网站中的自媒体:CaYPT等。

\item {} 
微信公众号:IYPT青年物理学家

\item {} 
\sphinxhref{https://www.wanmen.org/courses/586d23485f07127674135db9}{万门大学:青年物理学家竞赛}

\item {} 
\sphinxhref{https://www.icourse163.org/learn/UESTC-1003515005}{中国大学MOOC:物理学术竞赛挑战性课程}

\item {} 
任何一个学术搜索引擎,如百度学术

\item {} 
任何一个论文文档获取途径,如 \sphinxhref{https://sci-hub.org.cn/}{sci-hub}

\item {} 
Youtube(以及任何一个“科学上网”方式或者 \sphinxhref{https://www.onlinevideoconverter.com/video-converter}{onlinevideoconverter} )

\item {} 
Wikipedia,条目数量质量比百度好很多,常见的自然现象、本科以内的核心数理知识都能在上面找到系统的介绍:\sphinxurl{https://en.wikipedia.org}

\end{itemize}


\section{对参赛者的要求}
\label{\detokenize{2. Intro_Tournament:id7}}

\subsection{参赛前}
\label{\detokenize{2. Intro_Tournament:id8}}\begin{itemize}
\item {} 
有时间:竞赛的研究可能持续八个月及以上,尽管做研究的时间必然是一小部分,但完成研究任务所需的时间不可能很少。参赛者必须作出选择,放弃一些课余的其他活动。

\item {} 
有毅力:受挫、失败是科研的常态,必然不能畏难惧败。如果研究的结果都是可预料的,那就我们所做的就不能称为研究了。如果理论知识不可理解,那就再加阅读思考;如果实验现象不可预料,那就再加分析改进。

\item {} 
有动力:通常参赛者都未参与过物理方面的研究,不可能对这一过程有较充分的了解,更不能保证自己对其中的活动感兴趣。但你可能对现象本身感兴趣,深入了解一些物理以后又对在现象上应用物理规律感兴趣,完成一个研究任务后又从对现象的成功解释、预测中得到了成就感。不管动力是什么,只要能支持你前进就可以获得属于自己的丰富收获。

\end{itemize}


\subsection{参赛中}
\label{\detokenize{2. Intro_Tournament:id9}}\begin{itemize}
\item {} 
物理理论知识:没有物理理论知识,就没有从物理角度看待问题的思维方式,连提出有价值问题的能力都不具备,更别说解决实际物理问题还需要更多的具体分析方法。不同的题目需要不同的物理基础。

\item {} 
数学知识:物理学实质上是“自然哲学的数学原理”,数学手段的应用在物理中是必须的。微积分(包括常微分方程和级数等知识)是必须的,线性代数和偏微分方程也是常用的。

\item {} 
数学软件或编程语言:用于辅助推导和进行大量计算(尤其是人工难以完成的数值计算)。为降低学习成本,只建议使用Mathematica。

\item {} 
学术演示文稿的制作与展示的准备。

\end{itemize}


\section{对参赛者的好处}
\label{\detokenize{2. Intro_Tournament:id10}}\begin{itemize}
\item {} 
体验和测试:体验研究过程,测试自己是否能够对其感兴趣。

\item {} 
心性的锻炼:完成一次马拉松式的任务无疑是一次好的对个人意志的考验和对工作方法的训练,尤其是当它是第一次的时候。

\item {} 
能力的培养:积累数理基础和技能作为硬实力,但更重要的收获其实是那些粗略理解了的概念,它们使你的视野广博。此外,为了一个专门的任务所集中学习的知识掌握得往往更为牢固,因为你知道如何联系实际。

\item {} 
学术的交流:在研究过程中或赛场表现时,与同龄人进行学术问题的交流和探讨,能使思维敏捷,使你更善于发现问题、善于理解他人,也能使你更适应这样的活动。

\item {} 
其他:如保研等竞争性评选的加分项、毕业要求中创新学分的评定

\end{itemize}


\section{参赛后的建议}
\label{\detokenize{2. Intro_Tournament:id11}}
研究等方面的技术性的建议将在后面给出,这里只讨论技术之外的事情。
\begin{quote}
\begin{itemize}
\item {} 
如果你是大一学生,那么你的高等数学、线性代数等专业基础课也是要用心去学的,掌握更多的话对比赛也帮助相当大,在分析赛题时用到它们也能让你的这些课程学得更好。

\item {} 
适当记录你的研究进展,比如今天做了什么、得到了什么信息,不然在这样长的时间跨度下你可能会忘掉一些很有用的信息。

\item {} 
成功的参赛者通常为这个比赛付出了很多心血,但这并不意味着参赛者需要牺牲许多。把关心的问题时刻记在心头往往比单纯调用更多的时间更有用。

\item {} 
如果对比赛的制度、形式感到不满,不必仅仅因此否定比赛的价值,你可以仅仅是在其中做你想做的。

\item {} 
永远不要气馁。一般来说,从竞赛取得的最重要的收获就是在你突破自己能力边界的过程中取得的。如果一个问题更难解决,那么一个“好的研究”的标准自然也会更低。

\end{itemize}

\sphinxstyleemphasis{这部分的内容还需改善,所以也向有经验者征集建议}
\end{quote}


\chapter{研究简介}
\label{\detokenize{3. Intro_Research:id1}}\label{\detokenize{3. Intro_Research::doc}}

\section{科学的原则和价值观}
\label{\detokenize{3. Intro_Research:id2}}
百度百科上说,科学是一个建立在可检验的解释和对客观事物的形式、组织等进行预测的有序的知识的系统。

其中,进行预测则是科学的任务,因为人类总是有各种各样的需求,而满足这些需求需要人类做出适当的选择,掌握“如果这样会怎样”的信息对这种选择的帮助是决定性的,这是人类进行科学研究的重要动机,也是为什么科学研究能够得到各种各样的支持。

科学研究中,一切科学命题的真伪应当是 \sphinxstylestrong{可检验的} ,这就是科学研究与其他知识探索活动的区别所在。不可检验论述的一个著名的例子是“卡尔萨根的喷火龙”:某人声称他的车库里有一条喷火龙,但它具有完美的“透明”特性,以至于任何人类能掌握的探测手段都无法探测到它。这似乎显然是无稽之谈,但你如何反驳它呢?你无法拿出任何证据证明这条龙不存在,因为搜集证据需要你能够探测它,但如果你能探测到它,你找到的就一定不是他所宣称的具有完美隐匿特性的喷火龙。

我们能做的仅仅是指出这不是一个科学的(可检验的)论断,然后无视这条陈述。物理学家泡利无意中说的一句话被视为 %
\begin{footnote}[1]\sphinxAtStartFootnote
实际上这是过度解读,泡利最初说这句话并非是为了批判不可检验的论断。
%
\end{footnote} 对这种论断的批判: “Not even wrong(还不如错了)”。 将研究所要获取的知识限制在可检测的范围内,可以避免对无意义的命题进行不可能实现的研究,把人类有限的精力放在对那些可知的问题的好奇之上。仅仅是了解这一点核心理念,你就能很大程度上免受各种玄学、伪科学让你产生的困扰。

知识包含的信息量是有所不同的,这也一定程度使得知识的有用程度是不同的,一类最有用的科学知识就是那些具有普遍性的规律。比起“明天一定会下雨”这条信息,一个能够较精确预测天气的可行方法要有用的多,毕竟后者所提供的信息本身就能够判断前者的真伪,不过也有许多情况使我们更需要前者。这些普遍的事实规律就是一个科学理论体系的基础,通常需要经过(相对来说)很严格的审视,掌握它们能够让我们了解事实之间的关系,允许我们从一些好的角度看待自然现象,进而加以分析得到我们想知道的。于是在科学研究中, \sphinxstylestrong{普遍而非平凡的} 结论通常是我们更想要的,是更有价值的。

举一些不同的断言为例:
\begin{itemize}
\item {} 
平凡而普遍的断言:一只羊要么是黑的,要么不是黑的。

\item {} 
不普遍而非平凡的断言:这只羊是黑的。

\item {} 
普遍而不平凡的断言:所有的羊都是黑的。

\end{itemize}

现在有一个问题:上例中的最后一条是可检验的吗?考虑到研究人员的能力,我们没法检验所有的羊的毛色,但我们只需找到一只不是黑色的羊即可否定这条断言,所以这个断言是不可证明但可证伪的。考虑另一条断言“地球上的所有位置都没有喷火龙”,它是不可证明但可证伪的,那么你为何相信“地球上没有喷火龙”呢?

实际上,我们往往仅仅是“姑且相信”它,如果真的发现喷火龙,我们理应纠正自己错误的信念。如果我们因为不可严格检验而不相信地球上有喷火龙也不相信没有喷火龙,我们将无法建立相关的知识,而很多时候我们宁愿相信其中之一。这时只需要使用“ \sphinxstylestrong{不完全归纳} ”,比如做如下的推理:

因为我去过的地区A,B,C……都没有喷火龙,所以我相信地球上没有喷火龙。

这种归纳难免令人感到牵强,所以我们还需要把可证伪性视为我们重要的原则,使我们能够纠正自己的错误。说到这里,我们就能发现:我们找不到严格的普遍“真理”,或者说科学研究给出的普遍论断都不是 \sphinxstyleemphasis{绝对} 的真理。我们仅仅是基于自己想要了解事实的愿望去姑且相信些什么,然后手持“可证伪性”的利剑严格地、反复地审视之,才使得科学研究的结果充分可信。如果有某物在世间独一无二,且如果要验证关于它的某个断言的真伪就必须破坏它,那么这个实验将无法重复,这就导致相关的断言没法经过审慎的检验来变得可信,所以 \sphinxstylestrong{实验的可重复性} 很重要,尤其是对于那些在空间和时间上都具有普遍性的断言(比如各种物理定律)至关重要:对这种断言的检验要求在任何地点、在任何时间做的检验都给出相同的结果。

许多时候人们放弃了审视,仅仅是盲目地相信各种断言,于是就可能作出这样的推理:已知有数次占卜灵验了,所以占卜的结果都能够成功预测事实。不过要具体而严谨地分析占卜是不是有效,就不在本文的讨论范围之内了,读者不妨试试去设计一些能够评价其有效性的实验方法。

总的来说,科学研究中涉及的论断必须是一定程度上可检验的,而其中有价值的结论必然是可靠的、非平凡的。至于什么样的结论是平凡的、什么样的是不平凡的,这就需要通过学习该领域相关的具体理论来了解。从中学习到现象的分类、分解方法等等之后方能知道,现象中的哪些部分是许多现象都具有的共性、哪些是这个现象的特性。


\section{科研过程}
\label{\detokenize{3. Intro_Research:id4}}
由于科学知识都是关于现实的,那么获取它们的直接手段就是实验了。如果你想知道如果你按下这个按钮会导致机器怎么样,就弄一台(最好是一批)这样的机器来试试,然后用各种你用得上的方法观察机器的响应。这就是一类最简单的实验。在各种各样的实际研究中,你可能需要测试更“不稳定”的对象,对它的反应的探测也可能很困难。前者比如测试炸药爆炸的效果——不太容易保证能弄到一批差别非常小的炸药,后者比如观测某个细胞的位置——至少你不借助其他工具、仪器是做不到的。

按照我所说的,科研似乎完全是实践性的工作,实际上并非如此。要用更好的角度、方法应对问题,就必须 \sphinxstylestrong{了解关于你的研究对象的普遍规律} ,比如如果你研究液体的运动,就应当了解流体力学理论。只有掌握系统的理论知识,才能以好的观点(或者说,不很糟糕的观点)看待问题 %
\begin{footnote}[2]\sphinxAtStartFootnote
这里说得有些绝对。原则上来说,你也可以自己思考这些问题,创造自己的理论,以自己的方法分析问题,但那样的做法对比赛而言没有优势。
%
\end{footnote} ,物理领域的研究尤其如此。

只要你有一个想弄清楚的问题,那么在对它的科学研究中所需做的,无非就是上述所说的:
\begin{itemize}
\item {} 
学习理论知识

\item {} 
建立实验装置:要确保装置足够接近你的设想

\item {} 
进行实验并观测结果:可能需要掌握一些仪器的使用方法

\item {} 
从结果信息中分析问题的答案

\end{itemize}

不过往往也有在实验前就通过理论分析给出一个可被实验检验的论述的。只不过不同领域中的不同问题的研究需要不同的知识、方法、技能,都需要在研究的过程中根据需求来积累。更具体地,在这个比赛中的具体研究步骤则通常包括:题意分析、预实验(重现现象、熟悉操作)、基础知识学习、文献调研和阅读、理论分析(给出要验证的断言)、实验测量(验证理论分析结果)。我将在接下来的章节中给出进行这些活动的一些建议。


\chapter{自我定位}
\label{\detokenize{4. ChooseYourSkill:id1}}\label{\detokenize{4. ChooseYourSkill::doc}}
完成一个有效的、专业的研究项目,对研究者的水平有不低的要求。但是这个比赛是多人组队进行的,如果每个人专精于某一方面磨练技艺,就容易达到这样的要求了。有的题目难点主要在理论,有的则基本上只能做计算和实验工作。自己擅长的领域之外的问题,可以请队友解释清楚。

我将给出一些参考的发展路线。

\sphinxtitleref{这一部分尚在开发,并且实际上我没有能力完成它,希望有人协助。}


\section{理论}
\label{\detokenize{4. ChooseYourSkill:id2}}
做理论的门槛往往是在数学,熟悉数学的话实际上物理是很好学的。


\subsection{建模者}
\label{\detokenize{4. ChooseYourSkill:id3}}

\subsection{分析者}
\label{\detokenize{4. ChooseYourSkill:id4}}\begin{itemize}
\item {} \begin{description}
\item[{数学}] \leavevmode\begin{enumerate}
\item {} 
微积分

\item {} 
复数

\item {} 
级数

\item {} 
矩阵

\item {} 
常微分方程

\item {} 
傅里叶变换(可选)

\item {} 
偏微分方程

\item {} 
特殊函数

\item {} 
概率与统计(可选)

\end{enumerate}

\end{description}

\item {} \begin{description}
\item[{物理中常见的数学问题}] \leavevmode\begin{enumerate}
\item {} \begin{description}
\item[{运动、分布、传播、扩散问题}] \leavevmode\begin{itemize}
\item {} 
数学背景:常微分方程、偏微分方程

\item {} 
物理背景:质点力学、连续体力学、电磁场论、流体力学

\end{itemize}
\begin{enumerate}
\item {} \begin{description}
\item[{振动理论}] \leavevmode\begin{itemize}
\item {} 
物理模型:谐振子、梁方程

\item {} 
重要概念:稳定性

\end{itemize}

\end{description}

\item {} \begin{description}
\item[{波动理论}] \leavevmode\begin{itemize}
\item {} 
物理模型:波动方程、菲涅尔-惠更斯原理

\item {} 
重要概念:干涉、衍射、菲涅尔近似、夫琅禾费近似

\end{itemize}

\end{description}

\item {} \begin{description}
\item[{转动理论}] \leavevmode\begin{itemize}
\item {} 
刚体定轴转动、欧拉转动理论

\end{itemize}

\end{description}

\end{enumerate}

\end{description}

\item {} \begin{description}
\item[{泛函极值问题}] \leavevmode\begin{itemize}
\item {} 
数学背景:变分法

\item {} 
物理背景:拉格朗日力学、几何光学

\end{itemize}

\end{description}

\item {} 
统计物理与量子力学……

\end{enumerate}

\end{description}

\end{itemize}


\section{实验}
\label{\detokenize{4. ChooseYourSkill:id5}}\begin{itemize}
\item {} 
传感器

\item {} 
集成专业仪器

\end{itemize}


\section{计算}
\label{\detokenize{4. ChooseYourSkill:id6}}
实际问题往往是足够复杂的,以至于没有什么希望获得解析的结果,这时计算就是主力了。由于熟悉数据结构,做计算者一般也兼任数据处理。
\begin{itemize}
\item {} 
编程语言
\begin{itemize}
\item {} 
数组、字符串等基本数据、数据结构的运算与结构操作

\item {} 
代数方程的数值解法

\item {} 
数值积分、微分方程数值解法

\item {} 
有限元法、有限差分法(可选)

\end{itemize}

\item {} 
商业软件

\end{itemize}


\chapter{信息搜集}
\label{\detokenize{5. GetInfo:id1}}\label{\detokenize{5. GetInfo::doc}}
资料乃至于各种各样的信息的搜集在科研中是关键的,这可能包括教材里的基本理论、论文里的新分析方法、社交媒体上的实验视频等等。搜集信息的基本原则是时刻知道自己需要的是什么,从而能判断信息对自己有没有用。现代搜集这些信息的主要方式是互联网,所以使用搜索引擎、提炼有用信息的能力是至关重要的,这里不对它们作介绍。


\section{基本理论的学习}
\label{\detokenize{5. GetInfo:id2}}
学习的内容分为两方面:基本物理理论和具体问题的解决方法。没有对基本物理理论的理解,也就没有以物理观点看待、分析问题的思维方式,从而 \sphinxstylestrong{决不可能} 做出“物理”的研究。没有对具体实用技术、方法的了解,则也不能完成对实际问题的研究。

基本理论的学习通常按照研究的需求借助教材完成。如果你不知道对你的问题进行较专业的研究需要学习哪些知识,向有经验的人请教。

学理论知识时要注意几点:
\begin{enumerate}
\item {} 
\sphinxstylestrong{选择一本一般认为是好的教材。} 什么是好的教材?一本不错的教材应该让满足学习条件(掌握了先修知识)的学生 \sphinxstylestrong{能够} 通过它来充分学会其内容。也就是说,只要有足够的耐心,能在没看明白的情况下再看一遍或者深入思考直至弄懂,读者必定能掌握其中知识。如果你不知道哪些书是好的教材,向有经验的人请教。

\item {} 
\sphinxstylestrong{根据需求选读。} 比如某本流体力学的第一章和第三章是讲基本理论,第四章是讲理想流体的运动,第五章讲粘性不可压缩流体,而你只研究接近理想流体的流体,那只就需看一、三、四章。如果发现文中用到了自己没学的知识再翻阅其他章节了解。如果你不知道对你的问题进行研究需要学习哪些知识,向有经验的人请教。

\item {} 
\sphinxstylestrong{先准备好先修知识再学习你所需要的知识。} 比如在没有微积分和牛顿力学的基础的情况下看流体力学的书几乎肯定会一头雾水。如果你不知道所需的先修知识有哪些,向有经验的人请教。

\item {} 
\sphinxstylestrong{遇到不明白的地方,采取多种方式搜集信息来弄懂。} 通常本科范围内的概念性的知识在英文维基百科上都有靠谱的解说(至少物理领域的知识是这样),普遍开设的课程如微积分、大学物理中的许多问题也在网络上有广泛讨论。进一步的了解可以寻求其他书籍、论文、学者的帮助,除非世上没人知道真相(或者没人把真相说出来),不然总会有办法弄明白。有些抽象的概念、关系不易理解,就寻找更多的实例(通常例题、习题中就会有)来帮助理解。

\end{enumerate}

把握以上四点,我想应该就能保证理论学习的效率和效用了。


\section{论文的搜集和阅读}
\label{\detokenize{5. GetInfo:id3}}
通常成体系的知识才会写成专著,或是在为了方便培养学生时编写教材。写文章和会议报告是学术界大范围交流知识的主要两种方式,要即时了解关于某问题的研究进展,必须直接阅读学术论文。

搜集论文包括两个步骤,一是通过有关研究对象的关键词来寻找有哪些相关的论文,二是根据这些论文的题目去获取论文文本。第一步通常依靠学术搜索引擎完成,比如 谷歌学术、百度学术等,也可以直接使用文献数据库(如知网、万方、维普、Web of Science)中集成的搜索引擎。第二步则通过数据库或者盗版文献获取途径(如 sci-hub )完成.

搜索论文时要注意几点:
\begin{enumerate}
\item {} 
首先对你要研究的问题有所了解,知道你研究的问题的关键特点有哪些,并且知道业界是如何给各种概念、实体命名的,这样才能知道用什么样的 \sphinxstylestrong{关键词} 去搜索。如果你很不了解这些,或许就需要先学习基本物理理论;如果只是有些小疑惑,可以上网搜索或请教熟悉这领域的人。

\item {} 
如果对一个问题的研究已经有一定规模,那么很可能会有 \sphinxstylestrong{学位论文(Thesis)和综述论文(Review)} ,它们会对它所用的分析方法的理论基础和该问题的研究现状都作较详细的介绍,对新手很重要。

\item {} 
前沿研究成果一般都发布在 \sphinxstylestrong{英文期刊} 上(除了学位论文), \sphinxstylestrong{中文期刊} 上大多是一些不那么有价值、手段并不很高明的结果,但许多时候后者对新手更有用。

\item {} 
通过 \sphinxstylestrong{标题} 和 \sphinxstylestrong{摘要(Abstract)} 了解文章完成的工作和所用的方法,进而判断你是否需要阅读这篇文章。

\item {} 
避免付费。

\end{enumerate}

还有一点经验性的捷径:许多看上去有趣的现象都源于 Physics Teacher(教学类), Physics Today(资讯类), Nature, Physical Review Letters(顶级期刊), Arxiv(预印本网站) 等的一篇简短文章,从那里开始寻找或许能找到出题者最初的参考,了解问题的本源定义,以及通过引证文献扯出一片相关文献。

当你认为一篇论文能够让你更好地了解这个现象,或者你正尝试使用的分析方法与文章中所用的有关,你就有必要读一读相应的论文。学术文章的结构通常包括摘要、引言(Introduction)、主体(可能包括理论分析、实验结果等多个小节)和结论(Conclusion),其中引言部分通常是介绍研究对象的应用场景、基本原理、研究现状。

对于读文章的方法我有几个可用的建议:
\begin{enumerate}
\item {} 
如果你英文水平不高,也 \sphinxstylestrong{不必畏惧英文文章} 。学术写作的习惯不同于文学写作,多用简单词、简单句。用词基本上是日常词汇和该领域的术语,对于这两种生词都只需要现场查含义代入语句理解即可,持续阅读一段时间后生词量就基本没多少了。相比之下,如果因为英文水平尚不高而错过大量有用的文章,可能更加费时费力。

\item {} 
阅读过程中做 \sphinxstylestrong{简洁的笔记} 来总结收集到的信息,有助于对它们的理解和后续的查阅。

\item {} 
如果文章研究的对象与你的相同,你可以尝试 \sphinxstylestrong{重复文中的部分工作} ,包括理论推导、数值计算或实验测量。

\end{enumerate}


\section{如何快速判断资料的价值}
\label{\detokenize{5. GetInfo:id4}}
实际上以下都是废话,因为看完这些话对研究并无实际帮助,做不到的还是做不到,该积累的还是要自己积累。以下内容仅仅是要告诉读者,这些可能被认为是无用之物的见闻也可以是很有用的。
\begin{enumerate}
\item {} 
在对标题、摘要的鉴别之外,也要在搜索过程中逐渐熟悉各常见期刊分别是专攻什么问题的。如Physical Review系列的各个期刊、Arxiv Physics的各区是哪个领域的,此外可以通过 \sphinxhref{https://www.letpub.com.cn/}{LetPub} 查询期刊的SCI分区、影响因子来判断其中文章的在领域中的相对水平,可以借此避开某些灌水期刊的雷区。不过,文章是哪个期刊的不是根本重要的,重要的是有没有对你的研究有帮助的文章。

\item {} 
对于论文,先浏览摘要引言,再浏览图表公式,最后稍微细看引言部分的背景介绍。图表中可能包含理论模型或是实验装置的示意图,也可能有该学科用于分析的经典图表,如力学的相图(Phase Portrait)和庞加莱截面图、凝聚态物理的相图(Phase Diagram)、或是该领域的某关键物理量的空间分布图等等,借此一眼就能看出来文章研究的问题、领域、方法。至于一些具体的点图曲线图,通过坐标轴的标注和图题可以获取一些信息。凭空看公式通常是无法理解的,因为其中的符号的物理意义往往需要联系上下文去理解,但有些领域有些一眼就能看出是什么的式子,因为它们常用而通用,有着标志性的符号规定,如光学的菲涅尔衍射积分、流体力学的纳维-斯托克斯方程,像拉格朗日量这一包含着力学模型全部信息的式子更是关键的。看引言的问题背景可以积累相关领域的见闻,以及确定是否有必要调查与这篇文章相似的文献。

\item {} 
善用页面搜索功能。比如你可能想要判断维基百科上的 \sphinxstyleemphasis{Mechanical Filter} 页面究竟是在说“对物体的过滤器”还是“力学信号的滤波器”,那么你可以在整个页面中搜索“ \sphinxstyleemphasis{Quality Factor} (品质因数)”一词。品质因数是对滤波器性能的关键评价方式,对滤波器的探讨几乎必然包含这个词,从而你可以借此判断页面的内容是关于什么的,而不需要尝试通读整篇文章。

\end{enumerate}

说到底还是见闻的积累,各个领域的常用词汇、分析方法、核心概念、命名习惯、工程需求、衍生产品、甚至过去的一些著名科学家曾工作过的领域(许多他们提出的小问题都被研究至今)都是有可能帮助你判断的信息。有了见闻,就能作更合理的猜测,合理的猜测足够多就可以对资料用处的大小进行有效的评价了。


\chapter{研究}
\label{\detokenize{6. Research:id1}}\label{\detokenize{6. Research::doc}}

\section{初步分析}
\label{\detokenize{6. Research:id2}}
对题意和原理的分析是必不可少的,以IYPT2018第五题为例:
\begin{quote}

When a drinking straw is placed in a glass of carbonated drink, it can rise up, sometimes toppling over the edge of the glass.

Investigate and explain the motion of the straw and determine the conditions under which the straw will topple.
\end{quote}

初步的翻译结果大致是这样的:当一支吸管放在一杯碳酸饮料中会上浮,有时会从杯壁上翻倒。研究并解释吸管的运动,确定吸管翻倒的条件。


\subsection{明确现象}
\label{\detokenize{6. Research:id3}}
题中说了放置(placed),那么究竟是怎么放置的?是斜着插入饮料、还是横在水面上、或者是竖着立在杯底?光看这句话是无法分辨的,三种都有可能。

后面说吸管可能从杯壁上翻倒(topple over the edge,翻译未必准确),什么样算是topple?是从杯子里翻出去、还是从树立在杯中央的状态开始倾倒在杯壁上?

稍微做一些实际的测试,或者联系实际的经验,就知道上述的第二种现象是常常发生的,而第一种似乎不可能,但我又提不出第三种合理的诠释。考虑到题中说topple的现象是有时(sometimes)发生的,那么似乎又不是在说第二种,因为一般的吸管放在饮料里通常都会靠在杯壁上,除非吸管相当粗而重从而很容易就能竖着立在杯底(也有可能竖直浮在杯中而不接触杯子)。

原则上来说,两种立意都说得过去,研究的对象只要是符合题意的、说得通的就都没有什么可以诟病的地方,但它们的研究 \sphinxstylestrong{价值} 可能不一样,有的现象平凡、有的稀奇,有的现象有重要的实际应用、有的则没有——不过在我们这个主要以生活附近的现象为主题的研究中,一般不会考虑实际应用的价值。

去年我们做这题时,在尝试阶段观察到了第一种现象:把吸管放进饮料,然后就靠在了杯壁上,一会儿之后吸管就翻出了杯子。这让我们感到很新奇:它是怎么做到的?相比之下,另一种立意显得有些无聊,没有意料之外的事情——意料之外的事情正是研究中通常最让我们感兴趣的。于是我们就决定研究这个翻出杯壁的现象了,最感兴趣的问题就是“吸管是怎么翻出去的”,对这个问题的解答也自然包括题目中的任务目标:解答“什么情况下吸管会翻出去”。


\subsection{理解原理}
\label{\detokenize{6. Research:id4}}
确定了研究的目标之后,就要定性地分析现象的原理。我们关心的是运动问题,那么自然就采用力学的观点,进行受力分析。

吸管可能受到的影响比较大的力有:弹力(支持力)、重力、浮力、吸管上附着的气泡提供的“附着浮力”、液体中运动造成的粘滞阻力、吸管一端与杯壁的摩擦力。实验观察表明运动是比较慢的,所以初步的分析中可以考虑粘滞阻力这个与速度正相关的力。现象发生时首先吸管被放入饮料然后倒在杯壁上,一端伸出杯子、一端抵在杯底的边缘上,此时力和力矩都平衡。当饮料中气泡积累到一定数量,附着在吸管上的气泡能够打破这一平衡,使吸管上浮并旋转。随着吸管上浮,浮力会减小,但吸管会伸出更多从而使得重力产生的角加速度增大。当吸管下端接近液面,还会有表面张力来阻碍吸管的下端离开液面,不过这个力通常不大,或许可以忽略。只要此时吸管已经伸出足够多,吸管就会翻出去。

由上分析可知,现象的发生需要多个过程成功发生:吸管在杯中稳定(没有气泡也能翻出去的话,就没有特殊性了)、气泡提供的力克服了重力和摩擦力等力使吸管上浮、吸管伸出更多以后仍能上浮(要求重力的增强比浮力的削弱更显著)、吸管下端达到液面后仍能翻出去(要求重力力矩足以克服摩擦力和表面张力的影响)。于是进一步定量理论分析的目标也就明确了,就是利用力学规律计算什么装置参量能够同时满足以上要求。

只要有理论知识,在上述分析之前就可以大致确定有哪些装置的力学性质是值得注意的:杯子的形状和尺寸、液体的黏度和密度、液体填充量、吸管的密度和长度、吸管与杯壁间摩擦力的强度、碳酸饮料的气泡产生速率和附着效率。了解原理以后还可以更明确地定性分析这些参量如何影响,这里不再赘述,读者不妨自己试试。


\section{正式研究}
\label{\detokenize{6. Research:id5}}
正式的研究中涉及的要领都是具体的,在此我只能提一些重要原则和值得特别注意的地方。


\subsection{实验}
\label{\detokenize{6. Research:id6}}
在之前的探索中,你可能已经搭建过简易的装置来观察现象。但如果需要进行精确的实验测量,装置的严谨搭建就需要格外注意了,唯一的改善方法就是不断思考、不断测试哪里可能有意料之外的疏漏影响了你研究结论的可靠性。在实验中,有几点值得特别注意:
\begin{enumerate}
\item {} 
根据你的需求来购买器材,不一定要因为某些器材容易得到就使用它们。如果你的研究方案在现有条件下不能直接实现,寻找更巧妙的方案或者换一种研究思路。如果你不知道自己的需求或没有一个研究方案,说明你还需要通过简单的探索和测试来收集信息,在那之后你才能走到这一步。

\item {} 
如果你打算忽略装置某处的不严谨之处,你最好通过理论或实验粗略估计这样做造成了多大的影响,并且把估计的结果作为证据记录下来。

\item {} 
有些领域(如光学)有实验装置搭建的一些规范,如果有就先加以了解并应用之。

\item {} 
搭建好实验装置之后,还要尝试进行一些你可能用得上的测量,从而在熟悉操作方法的同时了解装置搭建和测量的方案有没有什么不妥之处。

\item {} 
许多实用设备的功能都是复杂的(比如手机相机往往会默认地进行自动对焦和图像优化),可能有意料之外的机制影响了测量到的结果。要么选用更简单的设备(比如使用CCD代替手机拍摄),要么使用可控的设备(比如使用专业的摄影相机或者拍照软件)。

\end{enumerate}


\subsection{数据处理}
\label{\detokenize{6. Research:id7}}
对实验数据进行处理通常是要将其整理并可视化(看一张图得到信息的速度比看一张表格要快得多),或是计算某个评价指标(如统计上的线性相关系数)。
\begin{enumerate}
\item {} 
必须掌握数字修约方法和不确定度的有关知识。

\item {} 
如果数据处理的任务复杂而繁重,可以使用计算机技术自动化地进行数据的采集和处理。

\end{enumerate}


\subsection{理论分析}
\label{\detokenize{6. Research:id8}}
理论分析的目标可能有很多种,但最常见的目标就是要使用普遍的物理规律 \sphinxstylestrong{预测} 某个具体模型的某些物理量的变化和分布规律。特别的建议有:
\begin{enumerate}
\item {} 
当问题过于复杂而使得你无法前进时,可以做一些近似或者换一种分析方法。寻找物理模型是否有简化的余地,求助于其他的数学工具以及计算机。可以使用带有符号计算功能的数学软件帮助你进行你所不能完成的推导,比如尝试对微分方程求解析解。如果你需要挑选使用什么工具(技术、方法、软件等),可以自己搜集资料决定,也可以向有经验的人请教。首要原则是要能解决你的问题,其次是要更有效率。许多情况下使用较完善的现成工具是好的,这样的工具资料更多从而易于学习、功能更强而更容易解决问题:不要重复发明一遍轮子。

\item {} 
关于重新发明轮子的问题,实际上也要把握一个度。许多时候就参赛者有限的能力而言,能够理解、重复前人的结果就已经是不错的进度了,完成这一步之后再考虑改进往往是更好的。在对力学运动的研究中,你完全没有必要重新从介质界面力学出发去寻找新的表面张力公式,用现成的经验公式也并不很差。但对于某些你完全摸不着头脑的情况,比如对于某些完全不在你已学过的知识之内的现象,你还是很有学习其基本理论的必要的。

\item {} 
对于许多领域的问题,你能找到的参量之间并非独立,为了避免多余的研究、揭示现象的本质,应该对各个物理量进行“无量纲化”。

\item {} 
专注于这个现象特别的地方,忽视那些不必要的因素。最杰出的一类物理学家能够做出尽可能夸张的简化、近似,同时保证结论没有多少偏差(当然,前者成立后者不成立的情况就是败笔了)。

\item {} 
得到结果以后,为了保证没有疏漏,做一次彻底的检查是很好的。这也能帮助你整理思路。

\end{enumerate}


\subsection{结论}
\label{\detokenize{6. Research:id9}}
给出结论原则上说只需要实验结果的支持,但如果你能够通过实验验证理论模型的有效性,就能一定程度上支持那些未被严格验证的理论分析结果。结论应当是一系列有价值的论断,并且它们得到了证据的支持从而成为科学的知识。


\chapter{赛场表现}
\label{\detokenize{7. Tournament:id1}}\label{\detokenize{7. Tournament::doc}}

\section{比赛规则与策略}
\label{\detokenize{7. Tournament:id2}}
国赛的比赛规则详见官方文件,其中的核心规则将在此被简化后重述、讨论。

每轮比赛中,三支3\textasciitilde{}5人的参赛队伍轮流扮演正方(Report)、反方(Opposite)、评论方(Review),进行三场(阶段)比赛。在国赛/地区赛/省赛内,这样由多场比赛组成的一轮比赛通常会更换对手举行数轮。

每一场(阶段)比赛定时45分钟,具体流程如下:


\begin{savenotes}\sphinxattablestart
\centering
\begin{tabulary}{\linewidth}[t]{|T|T|T|}
\hline
\sphinxstyletheadfamily 
流程
&\sphinxstartmulticolumn{2}%
\begin{varwidth}[t]{\sphinxcolwidth{2}{3}}
\sphinxstyletheadfamily 限时/min
\par
\vskip-\baselineskip\vbox{\hbox{\strut}}\end{varwidth}%
\sphinxstopmulticolumn
\\
\hline
反方挑战
&\sphinxstartmulticolumn{2}%
\begin{varwidth}[t]{\sphinxcolwidth{2}{3}}
1
\par
\vskip-\baselineskip\vbox{\hbox{\strut}}\end{varwidth}%
\sphinxstopmulticolumn
\\
\hline
正方准备
&\sphinxstartmulticolumn{2}%
\begin{varwidth}[t]{\sphinxcolwidth{2}{3}}
1
\par
\vskip-\baselineskip\vbox{\hbox{\strut}}\end{varwidth}%
\sphinxstopmulticolumn
\\
\hline
正方报告
&\sphinxstartmulticolumn{2}%
\begin{varwidth}[t]{\sphinxcolwidth{2}{3}}
12
\par
\vskip-\baselineskip\vbox{\hbox{\strut}}\end{varwidth}%
\sphinxstopmulticolumn
\\
\hline
反方提问
&\sphinxstartmulticolumn{2}%
\begin{varwidth}[t]{\sphinxcolwidth{2}{3}}
2
\par
\vskip-\baselineskip\vbox{\hbox{\strut}}\end{varwidth}%
\sphinxstopmulticolumn
\\
\hline
反方准备
&\sphinxstartmulticolumn{2}%
\begin{varwidth}[t]{\sphinxcolwidth{2}{3}}
2
\par
\vskip-\baselineskip\vbox{\hbox{\strut}}\end{varwidth}%
\sphinxstopmulticolumn
\\
\hline
反方报告
&
\textless{} 3
&\sphinxmultirow{2}{15}{%
\begin{varwidth}[t]{\sphinxcolwidth{1}{3}}
13
\par
\vskip-\baselineskip\vbox{\hbox{\strut}}\end{varwidth}%
}%
\\
\cline{1-2}
正反方讨论
&
\textgreater{}10
&\sphinxtablestrut{15}\\
\hline
评论方提问
&\sphinxstartmulticolumn{2}%
\begin{varwidth}[t]{\sphinxcolwidth{2}{3}}
3
\par
\vskip-\baselineskip\vbox{\hbox{\strut}}\end{varwidth}%
\sphinxstopmulticolumn
\\
\hline
评论方准备
&\sphinxstartmulticolumn{2}%
\begin{varwidth}[t]{\sphinxcolwidth{2}{3}}
2
\par
\vskip-\baselineskip\vbox{\hbox{\strut}}\end{varwidth}%
\sphinxstopmulticolumn
\\
\hline
评论方报告
&\sphinxstartmulticolumn{2}%
\begin{varwidth}[t]{\sphinxcolwidth{2}{3}}
4
\par
\vskip-\baselineskip\vbox{\hbox{\strut}}\end{varwidth}%
\sphinxstopmulticolumn
\\
\hline
正方总结发言
&\sphinxstartmulticolumn{2}%
\begin{varwidth}[t]{\sphinxcolwidth{2}{3}}
1
\par
\vskip-\baselineskip\vbox{\hbox{\strut}}\end{varwidth}%
\sphinxstopmulticolumn
\\
\hline
打分与讨论
&\sphinxstartmulticolumn{2}%
\begin{varwidth}[t]{\sphinxcolwidth{2}{3}}
4
\par
\vskip-\baselineskip\vbox{\hbox{\strut}}\end{varwidth}%
\sphinxstopmulticolumn
\\
\hline
总计
&\sphinxstartmulticolumn{2}%
\begin{varwidth}[t]{\sphinxcolwidth{2}{3}}
45
\par
\vskip-\baselineskip\vbox{\hbox{\strut}}\end{varwidth}%
\sphinxstopmulticolumn
\\
\hline
\end{tabulary}
\par
\sphinxattableend\end{savenotes}

比赛规则并不给出对参赛者的任务的具体限定,之后我会尝试根据规则对每个环节需要做什么的问题进行分析。


\subsection{关于正方的规则}
\label{\detokenize{7. Tournament:id3}}
正方需要进行的活动有
\begin{itemize}
\item {} 
进行研究报告

\item {} 
与反方讨论

\item {} 
回答反方和评论方的问题

\item {} 
进行总结

\end{itemize}

官方给出的对正方的要求:正方就某一问题做陈述时,要求重点突出,包括实验设计、实验结果、理论分析以及讨论和结论等。

官方给出的对正方的评分标准有:
\begin{itemize}
\item {} \begin{description}
\item[{内容}] \leavevmode\begin{itemize}
\item {} 
物理的正确性

\item {} 
论据是否切题

\item {} 
科学方法的正确运用

\item {} 
实验、理论及其一致性

\item {} 
结论的说服力

\end{itemize}

\end{description}

\item {} \begin{description}
\item[{表达}] \leavevmode\begin{itemize}
\item {} 
思路清晰

\item {} 
公式和符号的正确解释

\item {} 
正确的模型,量纲的一致性

\item {} 
视频资料(现场实验,音频,视频)

\item {} 
表达清楚

\item {} 
正确的参考文献

\end{itemize}

\end{description}

\item {} \begin{description}
\item[{讨论}] \leavevmode\begin{itemize}
\item {} 
物理的正确性

\item {} 
论据是否切题

\item {} 
恰当及扎实的物理知识

\item {} 
客观的辩论

\item {} 
对反方异议的讨论

\item {} 
礼貌的态度

\item {} 
对评论方的回答和总结发言

\item {} 
评委的问题和回答

\end{itemize}

\end{description}

\end{itemize}


\subsection{关于反方的规则}
\label{\detokenize{7. Tournament:id4}}
反方需要进行的活动有
\begin{itemize}
\item {} 
提问正方

\item {} 
进行反方报告

\item {} 
与正方讨论

\item {} 
回答评论方的问题

\end{itemize}

官方给出的对反方的要求:反方就正方陈述中的弱点或者谬误提出质疑,总结正方报告的优点与缺点。但是,反方的讨论过程不得包括自己对问题的解答,只能就正方的解答展开讨论。

官方给出的对反方的评分标准有:
\begin{itemize}
\item {} \begin{description}
\item[{问题/表达}] \leavevmode\begin{itemize}
\item {} 
提问是否离题

\item {} 
物理的正确性

\item {} 
表达清楚易懂

\item {} 
指出正方的优缺点

\end{itemize}

\end{description}

\item {} \begin{description}
\item[{讨论}] \leavevmode\begin{itemize}
\item {} 
物理的正确性

\item {} 
论点是否切题

\item {} 
恰当及扎实的物理知识

\item {} 
礼貌的态度

\item {} 
讨论正方的报告内容

\item {} 
正确回答评论方与评委的提问

\end{itemize}

\end{description}

\end{itemize}


\subsection{关于评论方的规则}
\label{\detokenize{7. Tournament:id5}}
评论方需要进行的活动有
\begin{itemize}
\item {} 
提问正反方

\item {} 
进行评论方报告

\end{itemize}

官方给出的对评论方的要求:评论方对正反方的陈述给出简短评述。

官方给出的对评论方的评分标准有:
\begin{itemize}
\item {} 
是否切题

\item {} 
指出正/反方在物理上及辩论中的优缺点

\item {} 
给出本阶段赛的一个完整的评价

\item {} 
讨论报告中的事实并避免冲突

\end{itemize}


\subsection{主控基本策略}
\label{\detokenize{7. Tournament:id6}}
从规则可以看出:正方的定位就是 \sphinxstylestrong{通过报告和讨论来展示研究成果} ,对于听者而言,其地位是基本的。反方的定位是 \sphinxstylestrong{借助提问和讨论来搜集信息,以此评价、揭示、总结正方报告中的优缺点} ,听者可以通过反方的活动深入了解正方研究。评论方的定位是 \sphinxstylestrong{整理总结报告和讨论的结果并补充遗漏点} ,能使听者更明朗了解之前的内容,并补充一些对正方研究方向以外的话题的了解,从而对问题本身都有一个全局的了解。

那么,每个环节的主控应当做什么?
\begin{itemize}
\item {} 
正方报告展示实验设计、实验结果、理论分析,进行讨论、给出结论。

\item {} 
反方提问设在反方报告之前,通常是用于询问、确认一些疑点,以便之后能写进报告。

\item {} 
反方报告可以总结正方的研究思路,并指出其中各做法的优势和缺陷。

\item {} 
正反方讨论时,反方作为“异议者”控制着话题,反方提出质疑的对象,双方共同讨论研究的合理性和改进方法。

\item {} 
评论方提问同样是用于确认疑点,尤其是可能的遗漏点。

\item {} 
评论方报告整理总结报告和讨论的结果并补充遗漏点。

\item {} 
正方总结报告总结自己的研究思路,并根据之前的讨论和反方、评论方的报告总结出一些可取的改进方案。

\end{itemize}

下列 \sphinxstylestrong{规则中列出的要点} 对于主控队员、尤其是正方和反方而言都是 \sphinxstylestrong{相通} 的:
\begin{itemize}
\item {} 
物理理解、研究方法的正确性、结论的说服力是不能通过改进赛场表现策略来提高的,而是依赖于之前的积累。为数不多能改进的就是:在想不到问题的答案时坦白自己不知道,而不是临时拼凑出一个连自己都不相信的不可靠结果;在结论的说服力确实不强的时候讨论其缺陷所在以及改进方案,而不是以为自己辩解为出发点。

\item {} 
使论据切题、思路清晰、表达清楚、辩论客观等是进行学术交流者所应当具备的交流能力。练习这些能力的一个重要原则是:站在听者的角度评价自己的发言。只有切题了的陈述、分类了的话题、精确的表达才能保证听者理解。而态度客观的辩论使得你更接近“正确”。

\item {} 
解释公式及其中符号的含义、使用多媒体资料是符合展示的根本原则的,能使被展示的内容容易理解、使你想表达的真确地传达到听者脑海。

\item {} 
用语礼貌、列出参考文献不需要解释。

\end{itemize}

由于正反评三方都有在赛场上 \sphinxstylestrong{临时准备报告} 的需求(时间可能非常紧张),准备一个演示文稿的模版并预先向里面填充可能用得上的内容是必要的。也不要仅仅在给定的准备环节准备演示文稿,在听正方报告、反方报告、正反讨论、评论方报告的同时就随着自己的思考输入一些文字是很有用的。这样,在报告前只需要填充少量的文本、进行一些删除和粘贴就能完成准备,而不需要输入过多的文字、不需要操心格式。同时,打字速度也将决定你能临时放上多少内容,如果主控打字很慢,可以让队友帮忙输入。

如果之前不清晰的讨论造成了对正方的误会,正方可以在进行 \sphinxstylestrong{正方总结报告时} 澄清,但不必作无谓的辩解,那可能会让你准备的其他可靠的内容也变得看上去不可信。


\subsection{关于其他队员}
\label{\detokenize{7. Tournament:id7}}
在每一阶段的比赛中,每支队伍只能由一人主控发言,其他队员只能做协助工作,可以和主控队员交流,但不能替代主控队员进行陈述。

如何让非主控队员发挥作用是很关键的。通常,其他队员可以
\begin{itemize}
\item {} 
提出对物理问题的见解,比如对特定问题的回答、或者对某一问题的价值的强调

\item {} 
关注那些不需要对题目有太多了解的方面,如报告的规范性

\item {} 
将以上内容进行筛选后,与主控适度地交流

\item {} 
记录讨论的内容以供回顾

\item {} 
帮忙改演示文稿

\end{itemize}

要知道主控很忙,使用文本(小纸条)进行提示应当是主要的方式,只有非常紧急的消息才适合用声音传递给主控 %
\begin{footnote}[1]\sphinxAtStartFootnote
有人指出,IYPT和CUPT实际上都不允许使用声音交流!在CUPT和IYPT的规则文件上找不到这一条,但各届CUPT的手册上都会添加一条 \sphinxstyleemphasis{似乎} 是这个意思的陈述。此外,据说有CUPT裁判主席阻止非主控队员说话的案例。
%
\end{footnote} 。队伍可能需要一个网络连接跨越物理位置的障碍来快速传递文本,这将是非主控队员之间的主要交流方式。


\subsection{关于队伍}
\label{\detokenize{7. Tournament:id10}}
每轮对抗赛中,每人最多主控 \sphinxstylestrong{2} 次。一支队的全部比赛中,每人最多作正方主控 \sphinxstylestrong{3} 次。\sphinxfootnotemark[1]

每一阶段的成绩由多位裁判给出,由 \(((\text{最高分}+\text{最低分})/2+\text{其他分数})/(\text{裁判数}-1)\) 计算,即极端分数权重减半的平均分。每一轮的成绩由三个阶段(角色)的成绩给出: \(\text{正方得分}\times 3+\text{反方得分}\times 2+\text{评论方得分}\),其中正方得分所占权重可能因为进行过多拒绝降低。

在一支队伍的全部比赛中正方对于可供挑战的题目,总计可以拒绝 \sphinxstylestrong{3} 次而不被扣分,之后每拒绝一次则从正方的基础加权系数 3 中扣去 \sphinxstylestrong{0.2} ,这个惩罚作用于所有轮次的分数。累计拒绝 \sphinxstylestrong{6} 次,将不计名次,不参与评奖。%
\begin{footnote}[2]\sphinxAtStartFootnote
这些规则中的数值可能与当届比赛的真实情况不同,国赛、区域赛、省赛的取值往往也是不同的。
%
\end{footnote}

反方可以向正方挑战任何一道题目,但有以下情况除外:
\begin{itemize}
\item {} 
在本轮,曾被陈述过的题目(不会被解除)

\item {} 
在所有轮次中,正方曾(作为正方)拒绝过的题目

\item {} 
在所有轮次中,正方曾(作为正方)报告过的题目

\item {} 
在所有轮次中,反方曾(作为反方)挑战过的题目

\item {} 
在所有轮次中,反方曾(作为正方)报告过的题目

\end{itemize}

如果可供挑战的题目小于5道(这种情况很少),则上述限制按照倒序依次解除,直到可挑战题数为5。

最后一轮各阶段的题目由正方在此阶段开始时自选并公布,相同题目仍然不能在同一轮中重复出现,之前作为正方报告过的题目仍不能被选择。自选此题进行陈述后,不能在决赛报告此题。

每轮比赛前,队伍应当结合手上的对阵图和选题情况表分析可能会做那些题目的正/反/评,进行进一步的准备(比如回顾之前准备好的笔记),尤其是要考虑好反方挑战什么题目。

由于正方分数加权系数高、且高度依赖于赛前实验研究的积累,通常会有避免做得不好的题目被挑战的需要。除了使用有限次数的拒绝权以外,利用规则也有希望做到这一点:如果你们的队伍作反方在作正方之前,那么你可以挑战你想要“保护”的题目。如果挑战成功,作正方时你们不可能被挑战这题。不过,有时规则不允许你挑战这题。


\subsection{进攻性策略}
\label{\detokenize{7. Tournament:id14}}
以上策略都是基于“如何提高自己得分”的“防御性策略”,实际上你还可以适当使用“进攻性策略”来降低对手的得分。了解它不仅仅是为了攻击别人,也为了保护自己。

防守型策略的本质是 \sphinxstylestrong{以研究本身为讨论核心} ,对事不对人。进攻性策略则包括通过诱导性提问来尽可能地展现对手的错误认识之类的做法,使其不符合评分表上的要求。根据规则,可以发现常规的进攻性策略包括
\begin{itemize}
\item {} 
诱导对方作出错误陈述。例:发现对方似乎对此领域的问题不了解之后,请对方谈谈对此领域某个基本物理概念的理解。应对方法:无,不要逞强以免进一步造成坏印象。要么就在赛前好好学习吧。

\item {} 
指出对方表述的不清晰:例:“你说的我不太明白,你的意思是不是XXXXX?”应对方法:重新清晰地表述,抵消之前表述混乱的影响。如果你做不到,那么或许意味着需要多练练口才、加深理解。

\item {} 
指出正方展示材料的格式不规范之处:例:发现正方没有在演示文稿上标注页码之后,请对方翻到某页,造成“正方材料不规范影响正常讨论”的印象。应对方法:使讨论继续进行,不受材料缺陷的影响,避免这一缺陷被强调。

\item {} 
诱导反方提及自己的实验。例:“我们就是这样的结果,你们做出不一样的结果了?”应对方法:以规则不允许为由拒绝直接回答,仅仅从理论上分析对方实验结果的合理性,避免实验结果的比较。

\end{itemize}

有效的进攻能让人无法确认自己在进行进攻,避免造成自己“卑鄙”的印象。此外,由于比赛的主题是物理,直接指出对方格式上的错误等问题属于离题行为,间接指出的方法可以参考上面的例子。

另一类通用的策略是通过反问、冗长的陈述、套话、扯开话题来 \sphinxstylestrong{扰乱对手思路、拖延时间} ,这在减少自己表现内容的同时阻止对手表现,甚至有可能引起对手情绪的波动,进而作出不审慎的思考和陈述。这个做法完全违背比赛初衷,我是痛恨的,如果你见过一次你大概也会痛恨它。一个比较友好的变体是:承认自己不了解对方希望讨论的内容从而拒绝讨论,这个做法起到类似的效果,但像是无奈之举,对自己的分数也有不小影响(这也是为什么正方弱的场次通常反方的得分也不高——讨论不起来)。应对的方式是:如果对手拉开话题,那么指出“话题正被拉开”这一点并询问对方是否要放弃当前话题的讨论;如果对手进行大量的反问,那么简短陈述自己的回答再反问回去;如果对手进行冗长的陈述,那么可以以时间有限为由打断对方(如果这个回答不那么重要),此后不要再提不关键的问题以避免对手有太多拖时间的机会。

进攻行为除了在比赛开始后降低对方得分的做法,还可以 \sphinxstylestrong{在要挑战的题目上进行精细的挑选} ,迫使对方做准备不良的报告或者消耗拒绝权。详细调查、分析对手学校的弱项的行为比较少见(我猜的),但常有挑战那些“公认难度较高”的题目的做法出现。我曾遇到过这样的情况:同一轮赛场上的三个队伍互相挑战某三道“难题”,就这三道题就消耗了三支队伍一共六个拒绝权!这种进攻不仅能由反方发起,在正方自选题目的最后一轮还能由正方发起,但准备不充分对反方的影响并不很大——至少没有对正方的大,而且通常实验研究上的缺陷比较显眼,而反方并不会提及自己的实验。不过,在末轮选“难题”的做法对于创新发明类型的第一题有特别的优势,因为正方提出的方案很可能是反方不曾想到的,从而使反方、评论方因为缺乏这方面的深入了解而难以发挥。

对于这类进攻没有很好的临时应对措施,唯有在使用拒绝权时更谨慎些,仅用它保护真正的弱点——我们在规划拒绝权时都是 \sphinxstylestrong{作最坏的打算} 。如果即便如此打算你们也需要碰运气才能使得自己不完全失去竞争力(过多拒绝而降低了不少加权分数、或者在某一场上表现极差),那就只能怪罪于之前做的研究数目不够了。

这里顺便讨论一下“接受对做得较差的题目的挑战”和“拒绝挑战而降低加权分数”两者之间的选择的优劣。一般的分析似乎表明:如果当前总分很高,那么降低加权分数的影响将会是显著的,这时选择前者;如果分数很低且做得好的题目未被挑战到,则可以考虑后者。实际上,任何情况下我都建议选择前者,因为你的得分是一门“玄学”——它所依赖的你的表现、裁判、对手等不确定性因素都难以预料,以致于你难以保证做得好的题目一定能拿高分。


\section{学术报告}
\label{\detokenize{7. Tournament:id15}}
进行学术报告就是要在有限的时间内把你的研究成果展示出来,要做好也是一门学问,这里仅仅提一些基本的要求:
\begin{itemize}
\item {} 
通常,好的讲述要么经过多次演练,要么讲者对讲述内容(研究的思路和细节)有深入的理解。

\item {} 
通常需要演示文稿作为一个列提纲展示思路、放图表展示证据的工具。宜简洁,避免有不必要的元素吸引观众的注意力。

\item {} 
给出的图表应当含义清晰,标注有表头、曲线颜色的含义、坐标轴代表的物理量等等,否则此图往往会无法提供任何可以理解的信息

\item {} 
一类严格的内容安排思路是:
\begin{itemize}
\item {} \begin{description}
\item[{立意陈述与研究方法简介}] \leavevmode\begin{enumerate}
\item {} 
题目介绍与分析

\item {} 
表明要弄清楚的对象 %
\begin{footnote}[3]\sphinxAtStartFootnote
一定要事先思考你的题意理解是不是主流的、是不是可被默认的,否则你就必须现场详细阐述你的理解,否则听众将以他们原有的印象理解乃至误解你的目标和动机、误解你所用的词汇的含义。
%
\end{footnote}

\item {} 
简述采用的研究方法

\end{enumerate}

\end{description}

\item {} \begin{description}
\item[{具体问题的分析与实证}] \leavevmode\begin{enumerate}
\item {} 
提出问题

\item {} 
理论分析

\item {} 
实验方案设计

\item {} 
实验数据分析

\item {} 
理论-实验比较

\item {} 
得出结论并评价理论模型和实验方案的合理性(比如:分析近似带来的理论计算误差、实验测量方法存在的系统误差等)

\item {} 
如果还研究了其他问题,则回到1.重复以上步骤

\end{enumerate}

\end{description}

\item {} \begin{description}
\item[{工作成果总结}] \leavevmode\begin{enumerate}
\item {} 
给出所有可靠的结论

\item {} 
对于不可靠、有矛盾的结论,从理论模型和实验方案着手解释,并尝试提出改进方案

\end{enumerate}

\end{description}

\end{itemize}

\item {} 
另一类“讲故事”风格的思路是:
\begin{enumerate}
\item {} 
给出题目

\item {} 
介绍实验装置和直观的实验现象

\item {} 
指出其中特别之处

\item {} 
进行简易的理论分析并给出一个定性的猜想

\item {} 
进行精细的理论分析来预测结果

\item {} 
介绍改进了的实验装置并展示实验数据

\item {} 
比较理论-实验结果并解释其中差异的来源(也是在提出新的问题)

\item {} 
指出更特别的地方(如果存在),回到3.

\item {} 
总结探索的历程以及得到的结果

\end{enumerate}

\item {} 
关于语言风格特色……

\end{itemize}

\sphinxstyleemphasis{这部分仍在开发}


\chapter{附录}
\label{\detokenize{8. Appendix:id1}}\label{\detokenize{8. Appendix::doc}}

\section{2020年竞赛题目与试析}
\label{\detokenize{8. Appendix:id2}}
中文翻译和注解仅供参考。翻译的原则是:在中英文的分句能一一对应的前提下,标题尽量还原词汇的背景含义和艺术性(而可能与题目内容关系并不密切),现象描述部分尽量亲切地表述原文最可能要指代的现象(而可能导致含义变得狭窄),任务指定部分尽量保持原意(而不限定任何题意理解和研究的方向)。

\sphinxstylestrong{关键词} 是我很不愿意写的栏目,因为这样写总是过于片面,但让新手接触一些相关的名词是有意义的。

\sphinxstylestrong{从这里开始} 是“亲民”的栏目,从中可以获取对问题的第一印象。

\sphinxstylestrong{高手示范} 栏目的参考文献都有很大的价值。有些你不得不参考(如第九题),有些你最好参考一下(如第八题),有些你参考不了(如第七题)。

\sphinxstylestrong{特别提醒} 栏目是我对网络上已有的参考文献列表的补充,我认为其背后可能有有价值的问题,有余力者可以考虑。

\sphinxstylestrong{路线} 是我提供的一条思路。我指出的研究领域和方向很可能不正确或是不准确,且仅建议在完全不知所措时参考,否则会错过锤炼信息搜寻能力、积累见闻的机会。即便我说的是对的,也应当尽量不照搬,因为寻找自己认为有价值的问题去研究也是研究重要的一环。而对于很多物理背景浓厚的问题,能够重复前人的成果就已经算是足够出色了,所以不需要路线上的额外建议,研究者只需跟随已有的论文即可。

白嫖更多资料,可以访问:\sphinxurl{http://stemfellowship.org/iypt-2020-references-2/}。


\subsection{1. 自己发明 Invent Yourself}
\label{\detokenize{8. Appendix:invent-yourself}}
Design an instrument for measuring current using its heating effect.

设计一个通过电流热效应来测量电流的装置。

What are the accuracy, precision and limits of the method?

这种方法的准确性、精密性、局限性 %
\begin{footnote}[1]\sphinxAtStartFootnote
\sphinxstyleemphasis{limits} 一词也可能特指 \sphinxstyleemphasis{检出限(detection limit)} 等概念,但那样的话不应写复数形式。也可能指装置性能的理论极限,但那样的话不应与 \sphinxstyleemphasis{accuracy, precision} 并列。此处可以理解为对电路的影响这类的局限性。
%
\end{footnote} 如何?
\begin{itemize}
\item {} \begin{description}
\item[{关键词}] \leavevmode\begin{itemize}
\item {} 
电流传感器(仪器科学、电路学)

\item {} 
(固体的)焦耳效应(固体电子学)

\item {} 
发热功率的统计(热学)

\end{itemize}

\end{description}

\item {} \begin{description}
\item[{路线}] \leavevmode\begin{itemize}
\item {} 
电路与电热器件设计:根据电路中信号的类型,在尽量增大发热量的同时尽量降低发热对电路中信号的影响。

\item {} 
热功率测量:响应快(时间分辨率高)、低系统误差、易定标的热传感器。

\item {} 
分析你所采用的这类设计各方面的理论极限。

\end{itemize}

\end{description}

\end{itemize}


\subsection{2. 低调的瓶子 Inconspicuous Bottle}
\label{\detokenize{8. Appendix:inconspicuous-bottle}}
Put a lit candle behind a bottle. If you blow on the bottle from the opposite side, the candle may go out, as if the bottle was not there at all.

在瓶子后方放一个点燃的蜡烛,在瓶子的前方对瓶子吹气可能会导致蜡烛熄灭,如同瓶子根本不在那儿一样。

Explain the phenomenon.

解释这个现象。
\begin{itemize}
\item {} \begin{description}
\item[{关键词}] \leavevmode\begin{itemize}
\item {} 
一定边界条件下的圆柱体绕流问题(流体力学)

\end{itemize}

\end{description}

\item {} \begin{description}
\item[{从这里开始}] \leavevmode\begin{itemize}
\item {} 
宽德效应 \sphinxhref{https://en.wikipedia.org/wiki/Coandă\_effect}{Coandă Effect}

\item {} 
蜡烛是如何被吹灭的?

\end{itemize}

\end{description}

\item {} \begin{description}
\item[{高手示范}] \leavevmode\begin{itemize}
\item {} 
M Trancossi, A Dumas, D Vucinic. \sphinxhref{https://www.researchgate.net/publication/289830815\_Mathematical\_Modeling\_of\_Coanda\_Effect}{Mathematical Modeling of Coanda Effect.} SAE Technical Papers, 2013, 10.4271/2013-01-2195.

\end{itemize}

\end{description}

\item {} \begin{description}
\item[{路线}] \leavevmode\begin{itemize}
\item {} 
进行对不同瓶子的不同气流流速、口径、距离、角度的定常流场分析,并借此预测吹灭蜡烛的难易程度。

\item {} 
瞬态模型

\end{itemize}

\end{description}

\end{itemize}


\subsection{3. 摇摆声管 Swinging Sound Tube}
\label{\detokenize{8. Appendix:swinging-sound-tube}}
A Sound Tube is a toy, consisting of a corrugated plastic tube, that you can spin around to produce sounds.

声管是一种由一根塑料波纹管构成的玩具,旋转 %
\begin{footnote}[2]\sphinxAtStartFootnote
原文 \sphinxstyleemphasis{spin} 似乎强调绕质心的自转是现象的关键,但标题 \sphinxstyleemphasis{swinging} 似乎强调着转动是以管的一端为瞬心的。
%
\end{footnote} 它就能产生声音。

Study the characteristics of the sounds produced by such toys, and how they are affected by the relevant parameters.

研究这样的玩具所发出的声音的特性,以及这些特性是如何被相关参量所影响的。
\begin{itemize}
\item {} \begin{description}
\item[{关键词}] \leavevmode\begin{itemize}
\item {} 
声驻波(声学)

\item {} 
惯性力引起的弯曲(连续体力学)

\item {} 
惯性力引起的气流(流体力学)

\end{itemize}

\end{description}

\item {} \begin{description}
\item[{从这里开始}] \leavevmode\begin{itemize}
\item {} 
\sphinxhref{https://en.wikipedia.org/wiki/Whirly\_tube}{Whirly Tube}

\item {} 
Amorim M E , Sousa T D , Carvalho P S , et al. \sphinxhref{http://sci-hub.tw/10.1119/1.3628265}{A Simple Experiment to Explore Standing Waves in a Flexible Corrugated Sound Tube{[}J{]}.} Physics Teacher, 2011, 49(6):360.

\end{itemize}

\end{description}

\item {} \begin{description}
\item[{路线}] \leavevmode\begin{itemize}
\item {} 
旋转中的波纹管形状

\item {} 
旋转中的波纹管中的气体流动

\item {} 
波纹管中的声音

\end{itemize}

\end{description}

\end{itemize}


\subsection{4. 唱歌的铁芯 \sphinxfootnotemark[3] Singing Ferrite}
\label{\detokenize{8. Appendix:singing-ferrite}}%
\begin{footnotetext}[3]\sphinxAtStartFootnote
原文Ferrite应译为铁氧体。硬磁铁氧体用于作为磁铁,而软磁铁氧体用于作为磁珠(铁芯),但它们都是铁磁性的,除矫顽力不同之外无根本不同。考虑铁氧体一词在生活中不常用,故根据题意以“铁芯”这一有代表性的印象代替(但实际铁芯确实是片状硅钢制成的)。
%
\end{footnotetext}\ignorespaces 
Insert a ferrite rod into a coil fed from a signal generator. At some frequencies, the rod begins to produce a sound.

在由信号发生器馈电的线圈中插入一根铁氧体棒,在某些频率下这根棒会发出声音。

Investigate the phenomenon.

探究这个现象。
\begin{itemize}
\item {} \begin{description}
\item[{关键词}] \leavevmode\begin{itemize}
\item {} 
磁声换能器(仪器科学)

\item {} 
磁致伸缩效应(固体物理学)

\end{itemize}

\end{description}

\item {} \begin{description}
\item[{从这里开始}] \leavevmode\begin{itemize}
\item {} 
磁致伸缩效应 \sphinxhref{https://en.wikipedia.org/wiki/Magnetostriction}{Magnetostriction}

\item {} 
Selim Yürekten, Yunus Sert, Miroslav Trnan, Enis Ceylan. \sphinxhref{http://www.sciencedirect.com/science/article/pii/S1877705817342534}{The Parameters of Generated Sound Level of Transformer Cores.} Procedia Engineering, 2017, 202:273-279.

\end{itemize}

\end{description}

\item {} \begin{description}
\item[{路线}] \leavevmode\begin{itemize}
\item {} 
换信号:谐振频率下的现象、窄带信号的频率响应特性、宽频带信号下的特性

\item {} 
换材料:各向同性/各向异性铁磁介质

\end{itemize}

\end{description}

\end{itemize}


\subsection{5. 甜蜜泡影 Sweet Mirage}
\label{\detokenize{8. Appendix:sweet-mirage}}
Fata Morgana is the name given to a particular form of mirage. A similar effect can be produced by shining a laser through a fluid with a refractive index gradient.

\sphinxhref{https://wikipedia.sogou.se/wiki/摩根勒菲}{摩根勒菲} %
\begin{footnote}[4]\sphinxAtStartFootnote
在关于亚瑟王的西方神话传说中,摩根勒菲是一名女性巫师的名字,也用于指代她用巫术所创造的空中城堡幻象。
%
\end{footnote} 是蜃景的一种特殊形式的名称。类似的现象 %
\begin{footnote}[5]\sphinxAtStartFootnote
看上去, \sphinxstyleemphasis{similar} 一词相当含糊,仿佛意味着各种各样的蜃景都可以是研究对象。实际上,狭义的 \sphinxstyleemphasis{Fata Morgana} 仅指某一类复杂蜃景,它相似于一般的上现蜃景,但它在竖直方向上包含多幅图像、或者说一系列图像)。
%
\end{footnote} 可以通过让一束激光通过折射率有梯度的流体来制造。

Investigate the phenomenon.

探究这个现象。

\noindent\sphinxincludegraphics{{5}.jpg}
\begin{itemize}
\item {} \begin{description}
\item[{关键词}] \leavevmode\begin{itemize}
\item {} 
折射定律(几何光学)

\item {} 
折射率梯度(一点点流体力学、热光效应等)

\end{itemize}

\end{description}

\item {} \begin{description}
\item[{从这里开始}] \leavevmode\begin{itemize}
\item {} 
Young A T , Frappa E . \sphinxhref{http://sci-hub.tw/10.1364/ao.56.000g59}{Mirages at Lake Geneva: the Fata Morgana{[}J{]}.} Applied Optics, 2017, 56(19):G59.

\end{itemize}

\end{description}

\end{itemize}


\subsection{6. 萨克逊碗 Saxon Bowl}
\label{\detokenize{8. Appendix:saxon-bowl}}
A bowl with a hole in its base will sink when placed in water. The Saxons used this device for timing purposes.

一个底部有洞的碗在水中会下沉,萨克逊人曾使用这装置计时。

Investigate the parameters that determine the time of sinking.

探究决定下沉时间的参量。
\begin{itemize}
\item {} \begin{description}
\item[{关键词}] \leavevmode\begin{itemize}
\item {} 
浮力、泊肃叶流动(流体力学)

\item {} 
动力学方程的逆函数(力学)

\end{itemize}

\end{description}

\item {} \begin{description}
\item[{从这里开始}] \leavevmode\begin{itemize}
\item {} 
任何一个实验视频

\item {} 
\sphinxhref{https://physics.stackexchange.com/questions/416112/what-physics-equation-can-be-used-to-describe-the-saxon-bowl-experiment}{StackExchange Physics: What physics equation can be used to describe the saxon bowl experiment?}

\item {} 
\sphinxhref{http://ilinblog.ru/article.php?id\_article=58}{Ilinblog: Saxon Bowl IYPT 2020 - analytical solution}

\item {} 
Greer A , Kincanon E . \sphinxhref{https://www.researchgate.net/publication/239045627\_An\_experiment\_with\_Saxon\_bowls}{An experiment with Saxon bowls{[}J{]}.} Physics Teacher, 2000, 38(2):112-112.

\end{itemize}

\end{description}

\item {} \begin{description}
\item[{路线}] \leavevmode\begin{itemize}
\item {} 
通过流体力学确定小孔流量的表达式

\item {} 
受力分析给出动力学方程,并将时间看作位置的函数积分

\end{itemize}

\end{description}

\end{itemize}


\subsection{7. 绳上球 Balls on a String}
\label{\detokenize{8. Appendix:balls-on-a-string}}
Put a string through a ball with a hole in it such that the ball can move freely along the string. Attach another ball to one end of the string. When you move the free end periodically, you can observe complex movements of the two balls.

在绳上串一个能自由移动的球,再固定一个球在绳的末端。当你周期性地移动另一末端,就能观察到两个球的复杂运动。

Investigate the phenomenon.

探究这个现象。

\noindent\sphinxincludegraphics{{pendulum}.jpg}
\begin{itemize}
\item {} \begin{description}
\item[{关键词/从这里开始}] \leavevmode\begin{itemize}
\item {} 
\sphinxhref{https://en.wikipedia.org/wiki/Astrojax}{Astrojax}

\item {} 
参数振动(理论力学)

\item {} 
驱动支点的摆问题(非线性动力学)

\item {} 
双摆问题(非线性动力学)

\end{itemize}

\end{description}

\item {} \begin{description}
\item[{高手示范}] \leavevmode\begin{itemize}
\item {} 
Philip Du Toit. \sphinxhref{http://www.cds.caltech.edu/~marsden/wiki/uploads/projects/geomech/Dutoit2005.pdf}{The Astrojax Pendulum and the N-Body Problem on the Sphere: A study in reduction, variational integration, and pattern evocation.} Student Projects in Caltech: Marsden Research Projects, 2005.

\end{itemize}

\end{description}

\item {} \begin{description}
\item[{路线}] \leavevmode\begin{itemize}
\item {} 
建立数值计算用的模型,进行大量的现象观察和现象分类(实验和数值计算)

\item {} 
从理论上定性解释每类常见现象的机制和特征

\item {} 
选择几类能导致稳定运动模式的驱动方式,分析两球的运动和驱动的关系

\end{itemize}

\end{description}

\end{itemize}


\subsection{8. 皂膜筛子 Soap Membrane Filter}
\label{\detokenize{8. Appendix:soap-membrane-filter}}
A heavy particle may fall through a horizontal soap film without rupturing it. However, a light particle may not penetrate the film and may remain on its surface.

下落的重的颗粒可能穿过一个水平肥皂膜而不弄破它,但轻颗粒则可能穿不过去而留在其表面上。

Investigate the properties of such a membrane filter.

探究这样的一个膜状筛子的性质。
\begin{itemize}
\item {} \begin{description}
\item[{关键词}] \leavevmode\begin{itemize}
\item {} 
表面张力、润湿性(介质界面物理)

\end{itemize}

\end{description}

\item {} \begin{description}
\item[{高手示范}] \leavevmode\begin{itemize}
\item {} 
BB Stogin, L Gockowski, H Feldstein, H Claure, J Wang, TS Wong. \sphinxhref{https://advances.sciencemag.org/content/4/8/eaat3276}{Free-standing liquid membranes as unusual particle separators{[}J{]}.} Science Advances, 2018, 4(8):eaat3276.

\end{itemize}

\end{description}

\item {} \begin{description}
\item[{特别提醒}] \leavevmode\begin{itemize}
\item {} 
Zou J , Wang W , Ji C , et al. \sphinxhref{http://sci-hub.tw/10.1063/1.4986798}{Droplets passing through a soap film{[}J{]}.} Physics of Fluids, 2017, 29(6):062110.

\end{itemize}

\end{description}

\end{itemize}


\subsection{9. 磁悬浮 Magnet Levitation}
\label{\detokenize{8. Appendix:magnet-levitation}}
Under certain circumstances, the “flea” of a magnetic stirrer can rise up and levitate stably in a viscous fluid during stirring.

在特定条件下,粘性液体中的磁力搅拌子会在搅拌时升起并稳定悬浮。

Investigate the origins of the dynamic stabilization of the “flea” and how it depends on the relevant parameters.

探究搅拌子的动态稳定的起源以及这是如何依赖于相关参量的。
\begin{itemize}
\item {} \begin{description}
\item[{关键词}] \leavevmode\begin{itemize}
\item {} 
粘滞阻力(流体力学)

\item {} 
磁场分布(电磁场论)

\item {} 
动力学稳定性 %
\begin{footnote}[6]\sphinxAtStartFootnote
注意这里说的动力学稳定性(Dynamics Stability)与题目中的动态稳定(dynamic stablization)不是同一概念!
%
\end{footnote} (力学)

\end{itemize}

\end{description}

\item {} \begin{description}
\item[{从这里开始}] \leavevmode\begin{itemize}
\item {} 
\sphinxhref{https://physics.aps.org/synopsis-for/10.1103/PhysRevLett.121.064502}{Synopsis: Levitating in a Fluid}

\end{itemize}

\end{description}

\item {} \begin{description}
\item[{高手示范}] \leavevmode\begin{itemize}
\item {} \begin{enumerate}
\setcounter{enumi}{10}
\item {} \begin{enumerate}
\item {} 
Baldwin, J.-B. de Fouchier, P. Atkinson, et al. \sphinxhref{https://arxiv.org/pdf/1805.08608.pdf}{Magnetic Levitation Stabilized by Streaming Fluid Flows{[}J{]}.} Physical Review Letters, 2018, 121(6):064502-.

\end{enumerate}

\end{enumerate}

\end{itemize}

\end{description}

\item {} \begin{description}
\item[{路线}] \leavevmode\begin{itemize}
\item {} 
位于定点的搅拌子的运动分析

\item {} 
流场、磁场分布分析

\item {} 
动力学稳定性分析

\end{itemize}

\end{description}

\end{itemize}


\subsection{10. 导电线 Conducting Lines}
\label{\detokenize{8. Appendix:conducting-lines}}
A line drawn with a pencil on paper can be electrically conducting.

用铅笔在纸上画出的一条线是电导性的。

Investigate the characteristics of the conducting line.

探究这根导电的线条的特性。
\begin{itemize}
\item {} \begin{description}
\item[{关键词}] \leavevmode\begin{itemize}
\item {} 
伏安特性(固体电子学)

\item {} 
频率响应特性(固体电子学)

\end{itemize}

\end{description}

\item {} \begin{description}
\item[{特别提醒}] \leavevmode\begin{itemize}
\item {} 
Kurra N , Dutta D , Kulkarni G U . \sphinxhref{http://sci-hub.tw/10.1039/C3CP50675D}{Field effect transistors and RC filters from pencil-trace on paper{[}J{]}.} Physical Chemistry Chemical Physics, 2013, 15(21):8367.

\end{itemize}

\end{description}

\item {} \begin{description}
\item[{路线}] \leavevmode\begin{itemize}
\item {} 
了解不同铅笔的石墨含量,以及石墨密度对导电性的影响。

\item {} 
对不同铅笔测量不同温度下的阻抗特性,并将结果与某些固体电子论的预测比较。

\end{itemize}

\end{description}

\end{itemize}


\subsection{11. 漂移斑点 Drifting Speckles}
\label{\detokenize{8. Appendix:drifting-speckles}}
Shine a laser beam onto a dark surface. A granular pattern can be seen inside the spot. When the pattern is observed by a camera or the eye, that is moving slowly, the pattern seems to drift relative to the surface.

向暗表面 %
\begin{footnote}[7]\sphinxAtStartFootnote
要求是暗表面可能只是为了保护眼睛和传感器,而并非是物理上的要求。
%
\end{footnote} 上照一束激光,在光斑内部可以看到颗粒状图案。用人眼或相机观察时,它是缓慢运动着的,看着就像图案在相对表面运动一样。

Explain the phenomenon and investigate how the drift depends on relevant parameters.

解释此现象并探究漂移是如何依赖于相关参量的。
\begin{itemize}
\item {} \begin{description}
\item[{关键词}] \leavevmode\begin{itemize}
\item {} 
漫反射(几何光学、统计光学)

\item {} 
干涉(波动光学)

\end{itemize}

\end{description}

\item {} \begin{description}
\item[{从这里开始}] \leavevmode\begin{itemize}
\item {} 
\sphinxhref{https://www.zhihu.com/question/27062939/answer/35097037}{知乎:为什么激光光束(或其反射)会看起来有颗粒感?}

\end{itemize}

\end{description}

\item {} \begin{description}
\item[{高手示范}] \leavevmode\begin{itemize}
\item {} 
Butters J N . \sphinxhref{https://libgen.pw/item/detail/id/969679}{Laser Speckle and Related Phenomena{[}J{]}.} Optica Acta International Journal of Optics, 1976, 23(10):842-843.

\end{itemize}

\end{description}

\end{itemize}


\subsection{12. 多边形旋涡 Polygon Vortex}
\label{\detokenize{8. Appendix:polygon-vortex}}
A stationary cylindrical vessel containing a rotating plate near the bottom surface is partially filled with liquid. Under certain conditions, the shape of the liquid surface becomes polygon-like.

一个静止圆柱管的底部是一个转盘,内部空间中有一部分填充着液体。在特定条件下,液体的表面会变得像多边形一样。

Explain this phenomenon and investigate the dependence on the relevant parameters.

解释这个现象并探究其与相关参量的依赖关系。

\noindent\sphinxincludegraphics{{12}.jpg}
\begin{itemize}
\item {} \begin{description}
\item[{关键词}] \leavevmode\begin{itemize}
\item {} 
旋转中的液体(流体力学)

\item {} 
流体力学对称性破缺(场论)

\end{itemize}

\end{description}

\item {} \begin{description}
\item[{高手示范}] \leavevmode\begin{itemize}
\item {} 
Jansson T R N , Haspang M P , Jensen K H , et al. \sphinxhref{https://arxiv.org/pdf/physics/0511251.pdf}{Polygons on a Rotating Fluid Surface{[}J{]}.} Physical Review Letters, 2006, 96(17):174502.

\end{itemize}

\end{description}

\item {} \begin{description}
\item[{路线}] \leavevmode\begin{itemize}
\item {} 
进行实验和数值计算,绘制相图(Phase Diagram)。

\item {} 
尝试给出关于液体表面形状的泛函极值形式规律,进而解释对称性破缺、计算“吸引域”。

\end{itemize}

\end{description}

\end{itemize}


\subsection{13. 摩擦振子 Friction Oscillator}
\label{\detokenize{8. Appendix:friction-oscillator}}
A massive object is placed onto two identical parallel horizontal cylinders. The two cylinders each rotate with the same angular velocity, but in opposite directions.

一个重物体 %
\begin{footnote}[8]\sphinxAtStartFootnote
原文 \sphinxstyleemphasis{massive object} 也可能指一般的有质量的物体,但考虑为较重的物体对本题的研究是有特别意义的。
%
\end{footnote} 放置在两根水平平行的相同圆柱上,而两根圆柱是以大小相同方向相反的角速度旋转着的。

Investigate how the motion of the object on the cylinders depends on the relevant parameters.

探究圆柱上物体的运动如何依赖于相关参量。

\noindent\sphinxincludegraphics{{13}.gif}
\begin{itemize}
\item {} \begin{description}
\item[{关键词}] \leavevmode\begin{itemize}
\item {} 
滚动摩擦(摩擦学)

\item {} 
振动(力学)

\end{itemize}

\end{description}

\item {} \begin{description}
\item[{从这里开始}] \leavevmode\begin{itemize}
\item {} 
Enrique Zeleny. \sphinxhref{http://demonstrations.wolfram.com/TheFrictionOscillator/}{The Friction Oscillator.} Wolfram Demonstrations Project, 2013.

\end{itemize}

\end{description}

\item {} \begin{description}
\item[{路线}] \leavevmode\begin{itemize}
\item {} 
给出支持力与位置的关系

\item {} 
确定摩擦模型

\item {} 
动力学分析

\end{itemize}

\end{description}

\end{itemize}


\subsection{14. 下落的塔 Falling Tower}
\label{\detokenize{8. Appendix:falling-tower}}
Identical discs are stacked one on top of another to form a freestanding tower. The bottom disc can be removed by applying a sudden horizontal force such that the rest of the tower will drop down onto the surface and the tower remains standing.

把相同的圆盘摞起来以形成一个自立式的塔 %
\begin{footnote}[9]\sphinxAtStartFootnote
仅限定每层只有一个圆盘,而未限定具体的摆放位置。也就是说,你或许可以造一个斜塔。
%
\end{footnote} 。底部的圆盘能通过施加一个突然 %
\begin{footnote}[10]\sphinxAtStartFootnote
可能指仅在造成可见位移之前有力的作用,也可能只是指移出底部砖块耗时很短。
%
\end{footnote} 的水平力来去除,同时 %
\begin{footnote}[11]\sphinxAtStartFootnote
可能仅指落下的瞬间,故塔的上方被震倒的情况或许可以不作考虑。
%
\end{footnote} 保持塔的剩余部分立着坐落在桌面上。

Investigate the phenomenon and determine the conditions that allow the tower to remain standing.

探究此现象并确定使允许塔保持直立的条件。
\begin{itemize}
\item {} \begin{description}
\item[{关键词}] \leavevmode\begin{itemize}
\item {} 
摩擦(摩擦学)、碰撞(弹性力学)、静力学稳定性(静力学)

\end{itemize}

\end{description}

\item {} \begin{description}
\item[{路线}] \leavevmode\begin{itemize}
\item {} 
在不同的装置参数下进行实验观察,寻找与真实移除过程相符的物理过程模型。

\item {} 
假定一类外力的形式,并计算其参数对塔的某些物理量能造成的影响。

\item {} 
建立衡量塔直立的难易程度的方式,找出临界条件。

\end{itemize}

\end{description}

\end{itemize}


\subsection{15. 胡椒罐子 Pepper Pot}
\label{\detokenize{8. Appendix:pepper-pot}}
If you take a salt or pepper pot and just shake it, the contents will pour out relatively slowly. However, if an object is rubbed along the bottom of the pot, then the rate of pouring can increase dramatically.

仅通过摇动调料瓶来倒出内容物 %
\begin{footnote}[12]\sphinxAtStartFootnote
注意题目原文并未限定内容物是什么, \sphinxstyleemphasis{salt or pepper} 仅仅是用于对容器的说明。
%
\end{footnote} 是相对较慢的,如果有一个物体在罐子底部摩擦则能戏剧性地增大倾倒的速率。

Explain this phenomenon and investigate how the rate depends on the relevant parameters.

解释这个现象并探究这个速率如何依赖于相关参量。
\begin{itemize}
\item {} \begin{description}
\item[{关键词}] \leavevmode\begin{itemize}
\item {} 
通过小孔/漏斗的颗粒流问题(颗粒力学、颗粒材料的统计力学)

\item {} 
随机信号驱动(随机动力学)

\end{itemize}

\end{description}

\item {} \begin{description}
\item[{高手示范}] \leavevmode\begin{itemize}
\item {} \begin{enumerate}
\item {} 
Coniglio, A. Fierro, H. J. Herrmann, M. Nicodemi. \sphinxhref{https://b-ok.cc/dl/849000/e4a2af}{Unifying Concepts in Granular Media and Glasses: From the Statistical Mechanics of Granular Media to the Theory of Jamming{[}J{]}.} 2004.

\end{enumerate}

\end{itemize}

\end{description}

\end{itemize}


\subsection{16. 镍钛引擎 Nitinol Engine}
\label{\detokenize{8. Appendix:nitinol-engine}}
Place a nitinol wire loop around two pulleys with their axes located at some distance from each other. If one of the pulleys is immersed into hot water, the wire tends to straighten, causing a rotation of the pulleys.

在两个有一定轴距的滑轮上绕一根镍钛丝。如果将其中一个滑轮浸入热水,镍钛丝就会趋向于伸直而使得滑轮转动。

Investigate the properties of such an engine.

探究这样的一个发动机的性质。
\begin{itemize}
\item {} \begin{description}
\item[{关键词}] \leavevmode\begin{itemize}
\item {} 
形状记忆效应(固体物理)

\item {} 
发动机性能(一点点机械动力知识)

\end{itemize}

\end{description}

\item {} \begin{description}
\item[{路线}] \leavevmode\begin{itemize}
\item {} 
寻找/建立相变时的应力等物理量的表达式,用于计算转矩、转速。

\item {} 
与实验相对比,计算能量效率、最佳工作温度区间等性能指标。

\end{itemize}

\end{description}

\end{itemize}


\subsection{17. 纸牌 Playing Card}
\label{\detokenize{8. Appendix:playing-card}}
A standard playing card can travel a very long distance provided that spin is imparted as it is thrown.

如果使一张标准纸牌 %
\begin{footnote}[13]\sphinxAtStartFootnote
不同规则、不同地域的标准是有些不同的,但这对研究的意义并无影响,因为这个条件的存在只是为了把研究范围限定在可手持的纸牌的范围内。
%
\end{footnote} 自转起来,它就能飞越很长一段距离。

Investigate the parameters that affect the distance and the trajectory.

探究影响距离和轨迹的参量。
\begin{itemize}
\item {} \begin{description}
\item[{关键词/从这里开始}] \leavevmode\begin{itemize}
\item {} 
陀螺效应(刚体力学)

\item {} 
\sphinxhref{https://en.wikipedia.org/wiki/Drag\_(physics)}{空气阻力} (空气动力学)

\item {} 
\sphinxhref{http://large.stanford.edu/courses/2007/ph210/scodary1/}{The Aerodynamics and Stability of Flying Discs}

\end{itemize}

\end{description}

\item {} \begin{description}
\item[{路线}] \leavevmode\begin{itemize}
\item {} 
练绝技(笑)

\item {} 
分析飞行过程的角动量方向偏移量、方向偏转造成的压差阻力。

\item {} 
分析强阻力下的飞行。

\end{itemize}

\end{description}

\end{itemize}


\section{启发性问题}
\label{\detokenize{8. Appendix:id32}}
以下是一些通用的启发性问题,没有正确答案。要对研究的对象有较深的了解,可以试着对它们进行一定的思考、作出自己的回答。
\begin{itemize}
\item {} 
题中所描述的现象是什么?有多种理解方式吗?如果有,哪种现象是你感兴趣的?

\item {} 
题目指定的研究任务是否足够明确,以至于能直接告诉你要做什么?如果不能,你打算把它具体化为对什么问题的研究?

\item {} 
现象的原理是什么?属于哪个学科的研究范围?已有的研究做到什么程度了?

\item {} 
是否能用简单而基本的理论完成一些偏差不很大的预测?如果不能,应当采用什么样的分析方法或者物理模型?

\item {} 
你所重现的现象与题目中描述的现象有什么差别?是否完全实现了题中的描述?除此之外你还得到了什么额外的信息?

\item {} 
装置中有哪些参量是你能调整的?你能想到的参量之间是独立的吗?它们对现象有没有性质上的或者数量上的影响?

\item {} 
装置的各个实体/要素对现象有什么影响?有它什么样、没它什么样、有无替代品?

\item {} 
现象发生的条件是什么?什么情况能发生、什么情况不能?

\item {} 
系统有无(近似的)守恒量?如果有,它在装置的各部分间是如何“转移”的?

\end{itemize}

\sphinxstyleemphasis{这一部分还需改善,所以也向有经验者征集建议}


\section{较有用的软件}
\label{\detokenize{8. Appendix:id33}}
数学软件:Mathematica(更全能)、Matlab(更快的矩阵运算)

编程语言:Python(更简单的语法)、C++(更高的性能)、Arduino(能迅速上手的单片机编程语言)

仿真模拟:COMSOL(更全能)、Ansys系列(某些模块有更多的优化,如流体和弹性体)、Proteus(电路仿真)

数据处理:Excel(更方便)、Origin(更专业)、Tracker(对视频中的物体进行跟踪)

演示:Powerpoint(更通用)、LaTeX Beamer(更专业)
\begin{quote}

广告:在这个比赛中,你可以仅学习 \sphinxstylestrong{Mathematica} ,这样的话以上的其他软件都可以免了。当然如果你已经有Matlab等软件的使用经验,或者有特种的需求(如超高性能计算),就另说了。应用于IYPT赛题的案例可参考: \sphinxhref{https://www.zhihu.com/question/27834147/answer/207262440}{知乎:wjxway - Mathematica 到底有多厉害?}
\end{quote}

工程制图:Solidworks(主要3D)、AutoCAD(主要2D)



\renewcommand{\indexname}{索引}
\printindex
\end{document}