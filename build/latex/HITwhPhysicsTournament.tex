%% Generated by Sphinx.
\def\sphinxdocclass{report}
\documentclass[a4paper,10pt,english]{sphinxmanual}
\ifdefined\pdfpxdimen
   \let\sphinxpxdimen\pdfpxdimen\else\newdimen\sphinxpxdimen
\fi \sphinxpxdimen=.75bp\relax

\PassOptionsToPackage{warn}{textcomp}
\usepackage[utf8]{inputenc}
\ifdefined\DeclareUnicodeCharacter
 \ifdefined\DeclareUnicodeCharacterAsOptional
  \DeclareUnicodeCharacter{"00A0}{\nobreakspace}
  \DeclareUnicodeCharacter{"2500}{\sphinxunichar{2500}}
  \DeclareUnicodeCharacter{"2502}{\sphinxunichar{2502}}
  \DeclareUnicodeCharacter{"2514}{\sphinxunichar{2514}}
  \DeclareUnicodeCharacter{"251C}{\sphinxunichar{251C}}
  \DeclareUnicodeCharacter{"2572}{\textbackslash}
 \else
  \DeclareUnicodeCharacter{00A0}{\nobreakspace}
  \DeclareUnicodeCharacter{2500}{\sphinxunichar{2500}}
  \DeclareUnicodeCharacter{2502}{\sphinxunichar{2502}}
  \DeclareUnicodeCharacter{2514}{\sphinxunichar{2514}}
  \DeclareUnicodeCharacter{251C}{\sphinxunichar{251C}}
  \DeclareUnicodeCharacter{2572}{\textbackslash}
 \fi
\fi
\usepackage{cmap}
\usepackage[T1]{fontenc}
\usepackage{amsmath,amssymb,amstext}
\usepackage{babel}
\usepackage{times}
\usepackage[Sonny]{fncychap}
\usepackage{sphinx}

\usepackage{geometry}

% Include hyperref last.
\usepackage{hyperref}
% Fix anchor placement for figures with captions.
\usepackage{hypcap}% it must be loaded after hyperref.
% Set up styles of URL: it should be placed after hyperref.
\urlstyle{same}
\addto\captionsenglish{\renewcommand{\contentsname}{Contents:}}

\addto\captionsenglish{\renewcommand{\figurename}{图}}
\addto\captionsenglish{\renewcommand{\tablename}{表}}
\addto\captionsenglish{\renewcommand{\literalblockname}{列表}}

\addto\captionsenglish{\renewcommand{\literalblockcontinuedname}{续上页}}
\addto\captionsenglish{\renewcommand{\literalblockcontinuesname}{continues on next page}}

\addto\extrasenglish{\def\pageautorefname{页}}

\setcounter{tocdepth}{1}


\hypersetup{unicode=true}
\usepackage{CJKutf8}
\DeclareUnicodeCharacter{00A0}{\nobreakspace}
\DeclareUnicodeCharacter{2203}{\ensuremath{\exists}}
\DeclareUnicodeCharacter{2286}{\ensuremath{\subseteq}}
\DeclareUnicodeCharacter{2713}{x}
\DeclareUnicodeCharacter{27FA}{\ensuremath{\Longleftrightarrow}}
\DeclareUnicodeCharacter{221A}{\ensuremath{\sqrt{}}}
\DeclareUnicodeCharacter{221B}{\ensuremath{\sqrt[3]{}}}
\DeclareUnicodeCharacter{2295}{\ensuremath{\oplus}}
\DeclareUnicodeCharacter{2297}{\ensuremath{\otimes}}
\begin{CJK}{UTF8}{gbsn}
\AtEndDocument{\end{CJK}}


\title{HITwhPhysicsTournament Documentation}
\date{2018 年 09 月 14 日}
\release{}
\author{N518 asd1dsa}
\newcommand{\sphinxlogo}{\vbox{}}
\renewcommand{\releasename}{}
\makeindex

\begin{document}

\maketitle
\sphinxtableofcontents
\phantomsection\label{\detokenize{index::doc}}



\chapter{前言}
\label{\detokenize{Preface::doc}}\label{\detokenize{Preface:id1}}
我代表学校参加了第八届大学生物理学术竞赛(CUPT),并且之后带了一届学生参赛,并且本身算是一个物理爱好者,对这个竞赛相对还算比较了解,在此为各位介绍它。

这个比赛首先是一个 \sphinxstylestrong{学术竞赛} ,而非一般学科竞赛中以做题为考核方式。涉及的主要任务是研究,其次是展示。研究的思路和成果虽说通常与真正的科学研究尚有差距,但至少算是一次科研的体验之旅。

这个比赛通常是低年级本科生参加,高中毕业后至多不到两年的学生不太可能积累有较充分的知识储备,一切所需的理论知识和研究方法都需要自己钻研、磨练。经过这样一个在实践项目中学习的过程,对理论知识与实际的联系会更容易,也能培养提出问题、解决问题的重要思维习惯。

我认为,它最大的意义是作为一个尝试的机会。高中生进入大学就开始修习专门的专业知识,但往往在并不了解学科的情况下就茫然学习,通常无法建立真实的兴趣。要寻找到自己的兴趣,需要先去尝试各种各样的可能适合自己的活动,这样才能了解这些活动,自然地建立兴趣,进而拥有追求,便不再会为人生的抉择而犯难。而我认为,这个比赛就是科研这条路线的尝试的一个好机会。

值得一提的是,虽然其名为物理学术竞赛,实际上它对各理工科专业的学生都有很好的锻炼作用,反倒由于低年级学生难以积累过多物理知识而使得物理学理论中特有的经典思想方法不太会出现,许多物理系的学生也会因为理论知识的学习任务太重而无心参与这个比赛,于是工科思维在这个比赛的研究中也占据了相当的分量。不过这对于物理学也不是问题,物理学虽说是抽象的“自然哲学的数学原理”,但终归是一门科学,是讲究可证伪、讲究实际的,专注于解决问题的工程思维并不会成为学理论物理的障碍。

不过,我所说的这样的尝试的意义在于区分自己是不是能够对这个活动感兴趣,那就意味着你参加之前不可能知道自己做起来以后会不会感兴趣。

既然这样,那就来试试吧!

这和科研的一个特性很像:平时我们一但遇到了预料之外的、失去掌控的东西便会苦恼,但做研究是思路似乎就是“要迎难而上”,因为预料之外的、失去掌控的事物就是研究者所要寻找、理解的。

虽然你参加之前不可能知道自己做起来以后会不会感兴趣,但为了区分自己是不是能够对这个活动感兴趣就需要做这样的尝试。既然这样,如果你在乎自己是否真的适合理工科领域的科研,那就来试试吧。

你确实很容易就可以从中寻找到乐趣。这种乐趣和题目的设计也有关系:竞赛题目指定要研究的现象大多源于生活,往往我们早已遇到过它们但未曾深入思考、仔细调查。涉及的领域距离我们也并不遥远,不是天上的脉冲星也不是微观的纠缠粒子,而是身边的流体、光、声等等。它们往往乍一看匪夷所思,但稍作思考、测试,便能收集到更多的信息,我们就可能提出一些假想来解释它,经过一些区分能力更强的实验检验以后,我们最终可以相信其中一个答案,这时就会有柳暗花明之感洋溢而出。

此外,如果争取到代表学校参赛的资格,就能进行另一项重要的活动——与其他高校的学术辩论。学术辩论不同于普通的辩论,其目的到底还是清晰地展示和客观地评价一个研究工作,从而探讨更好的改进方案,其首要原则是实事求是,不能诡辩。赛场上面向的不是对手而是伙伴,共同的目标是弄清楚现象的物理机制,而真相只有一个,不存在两方观点相矛盾但又都有道理的情况,讨论的目的是找到那个真相而非言语上取胜,相互的质疑只是为了帮助消除个人思维的局限性罢了。

\%模块大纲:科研教学、物理教学、题目分析教学、实验教学、展示与辩论教学


\chapter{竞赛简介}
\label{\detokenize{Introduction::doc}}\label{\detokenize{Introduction:id1}}
规则和建议都暂且简单介绍,规则详见官方 \sphinxstylestrong{竞赛指南} ,建议则会在接下来的章节细化。


\section{竞赛任务}
\label{\detokenize{Introduction:id2}}
竞赛要求一个队伍(3\textasciitilde{}5人)完成指定的研究任务,然后在赛场上扮演 \sphinxstylestrong{正方{[}Report{]}} 、 \sphinxstylestrong{反方{[}Opposite{]}} 、 \sphinxstylestrong{评论方} 三种角色。三种角色的任务分别是 \sphinxstylestrong{报告研究成果并与反方讨论以及回答反方和评论方的问题} 、 \sphinxstylestrong{总结正方研究的优缺点并且深入讨论改进方案或物理本质等} 、 \sphinxstylestrong{总结报告和讨论要点并且补充遗漏点} 。


\section{赛题}
\label{\detokenize{Introduction:id3}}
赛题即需要完成的研究内容,直接使用IYPT的赛题,一共17道,需要完成其中的15道及以上(地区赛要求稍低),如果放弃题目会有分数上的惩罚。赛题的内容一般包括两部分: \sphinxstylestrong{现象描述} 和 \sphinxstylestrong{任务指定} ,偶尔也有不由它们构成的题目。


\subsection{常见的研究任务}
\label{\detokenize{Introduction:id4}}\begin{itemize}
\item {} 
原理解释: Explain …

\item {} 
创新制造: Construct …/Design …

\item {} 
研究参量对现象的影响: Investigate how … depends on …

\item {} 
最优化(为了最优化,一般也需要完成上一种任务): Optimize …

\item {} 
性能评价(许多时候被其他的任务所隐含)

\item {} 
寻找现象发生的条件: Determine the condition/parameters that …

\item {} 
自由选择: Investigate the phenomenon.

\end{itemize}


\section{竞赛主要相关资料来源}
\label{\detokenize{Introduction:id5}}\begin{itemize}
\item {} 
IYPT官方网站:iypt.org

\item {} 
GYPT官方网站:gypt.org(德文,建议翻译成英文看)

\item {} 
Reference Kit发布者网站: kit.ilyam.org

\item {} 
微信公众号:IYPT青年物理学家

\item {} 
任何一个学术搜索引擎,如百度学术

\item {} 
任何一个论文文档获取途径,如sci-hub

\item {} 
Youtube(以及任何一个科学上网方式)

\end{itemize}


\section{对参赛者的要求}
\label{\detokenize{Introduction:id6}}

\subsection{参赛前}
\label{\detokenize{Introduction:id7}}\begin{itemize}
\item {} 
有时间:竞赛的研究可能持续八个月及以上,尽管做研究的时间必然是一小部分,但完成研究任务所需的时间不可能很少。参赛者必须作出选择,放弃一些课余的其他活动。

\item {} 
有毅力:受挫、失败是科研的常态,必然不能畏难惧败。如果研究的结果都是可预料的,那就我们所做的就不能称为研究了。如果理论知识不可理解,那就再加阅读思考;如果实验现象不可预料,那就再加分析改进。

\item {} 
有动力:通常参赛者都未参与过物理方面的研究,不可能对这一过程有较充分的了解,更不能保证自己对其中的活动感兴趣。但你可能对现象本身感兴趣,深入了解一些物理以后又对在现象上应用物理规律感兴趣,完成一个研究任务后又从对现象的成功解释、预测中得到了成就感。不管动力是什么,只要能支持你前进就可以获得属于自己的丰富收获。

\end{itemize}


\subsection{参赛中}
\label{\detokenize{Introduction:id8}}\begin{itemize}
\item {} 
物理理论知识:没有物理理论知识,就没有从物理角度看待问题的思维方式,连提出有价值问题的能力都不具备,更别说解决实际物理问题还需要更多的具体分析方法。不同的题目需要不同的物理基础。

\item {} 
数学知识:物理学实质上是“自然哲学的数学原理”,数学手段的应用在物理中是必须的。微积分(包括常微分方程和级数等知识)是必须的,线性代数和偏微分方程也是常用的。

\item {} 
数学软件或编程语言:用于辅助推导和进行大量计算(尤其是人工难以完成的数值计算)。为降低学习成本,只建议使用Mathematica。

\item {} 
学术演示文稿的制作与展示的准备。

\end{itemize}


\section{对参赛者的好处}
\label{\detokenize{Introduction:id9}}\begin{itemize}
\item {} 
体验和测试:体验研究过程,测试自己是否能够对其感兴趣。

\item {} 
心性的锻炼:完成一次马拉松式的任务无疑是一次好的对个人意志的考验和对工作方法的训练,尤其是当它是第一次的时候。

\item {} 
能力的培养:积累数理基础和技能作为硬实力,但更重要的收获其实是那些粗略理解了的概念,它们使你的视野广博。此外,为了一个专门的任务所集中学习的知识掌握得往往更为牢固,因为你知道如何联系实际。

\item {} 
学术的交流:在研究过程中或赛场表现时,与同龄人进行学术问题的交流和探讨,能使思维敏捷,使你更善于发现问题、善于理解他人,也能使你更适应这样的活动。

\item {} 
其他:如保研等竞争性评选的加分项、毕业要求中创新学分的评定

\end{itemize}



\renewcommand{\indexname}{索引}
\printindex
\end{document}