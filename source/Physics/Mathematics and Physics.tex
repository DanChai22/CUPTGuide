% !TEX program = xelatex
\documentclass[10pt]{article}
\usepackage{NotesTeX2,lipsum,amsmath}

\renewcommand{\minalignsep}{0pt}
\renewcommand{\contentsname}{目录}
%\renewcommand{\figurename}{图}

\title{
	\begin{center}
		{\Huge 先修数学物理知识}%掌握某些必要而通用的物理知识对于解决各领域的物理问题都有很大的帮助,甚至其中的一些如果不掌握就会有大的危害。我将进行这样的几个专题的介绍。只要求读者有微积分、矩阵的概念性知识和计算能力。
	\end{center}
}
\author{N518 asd1dsa}
\affiliation{
哈尔滨工业大学(威海)\\
理学院
}

\begin{document}
	\maketitle
	\flushbottom
	\newpage
	\pagestyle{fancynotes}
	\part{定性与半定量分析}
		\section{对称性}
		\section{量纲}
		\section{经典与量子、宏观与微观、低速与高速}

	\part{分析}
		\section{偏微分方程的分离变量法}
		\section{二阶常微分方程}
		\section{近似}
		\section{微扰}

	\part{计算、实验、数据}
		\section{数值计算原理}
		\section{测量不确定度}
		\section{拟合}

\end{document}